\chapter{Introduction}

Todays technology offers alot of technical solutions to a variety of problems. 
Free and open source initiatives can provide software that before had a price tag.
Technology is ever getting cheaper and internet penetratation rates are ever increasing.
The technical requirements are being met. 
Therefore there are a puzzle to me why the technologies are not being taken advantage of.
It is well known that there should be alot inefficiensies that could be remedied by introducing \gls{ict}'s in different areas.
In the more developed countries of the world we have been able to apply \gls{ict}'s to some level of success.
In the less developed countries of the world \gls{ict} projects have a tendency to fail or the progress are not moving forward as fast as expected, should be.


Why is it then, that \gls{ict} have a hard time being applied when the oppertunities are already there?
If the technologies are available and ready to being taken to use. 

This thesis will try to give an answer to this question. 
With an action research study taken place in one of the countries that are currently trying to take advantage of the rapid technological advancements of \gls{ict}'s.
The \gls{ar} project executed in collaboration with an organization called \gls{hisp} who are currently trying to implement a \gls{is} system called \gls{dhis2} in several developing countries.
This may provide insiders perspective on the current challenges we are facing in this area. 
The action research focus primeraly on introducing \gls{ict}'s in the health sectior of developing countries where the need for improvement are most critical. 

Firstly we explore the current status of this area in the research community. 
E-health gives us an introduction to why we are trying to introduce \gls{ict}'s in order to improve health care. 
Work in transition strategies are relevant in the sense that these challenges have been met before when trying to upgrade current systems. 
Introducing \gls{ict}'s in developing countries has been attempted before with little success, but have resulted in experiences that can provide us with guidelines of how to avoid the same result. 

Further there is another motivation of introducing \gls{ict}'s in the developing countries that relates to better the economy of developing countries which introduces the "learning economy". 

Then move on to more broad contextual description of the case in order to get a sense of that there are interconnected dependencies and we are not dealing with a isolated situation. 
Before introducing the primary result of the study I try to elaborate on the method used.
With the case in context and an understanding of the method used I introduce the case as seen from my perspective. 

It builds on qulitative data collected from collaborating with professionals at the \gls{moh} in Rwanda. 
Combining research with everyday practise as is the main purpose of action research. 
In \gls{ar} there are interventions that will have an effect on the problem situation.
These interventions lead to the development of two applications that also are a part of this reasearch project. 
The experiences from this case study are therfore useful since the interventions of the project lead to concrete products and therefore successfully introduced \gls{ict} products in the health sector. 



With this we can then try to answer the following questions in the discussion and conclusion.

How can we then improve the process of introducing new technology in developing countries? and what are the necessary actions needed in order to increase the success rates of \gls{ict} initiatives?
Are there any new challenges that have not yet been highlighted as a result of this \gls{ar}-project?
In my opinion, the need to act on \gls{ict}'s as a way of improving health services in the developing countries are long overdue. 




