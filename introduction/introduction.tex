\chapter{Introduction}
\section{Philosophy}
I will try to take on both an interpretive and a positivist view of my research.
The reason I do this is that I believe that one does not exists without the other.
No one really have a shared reality, and is never completely different.
The positive researcher will concentrate on the shared knowledge in a community, while the interpretive will try to harmonize the different realities. The users of the knowledge I am trying to create are the academics, focused in the field of Information Systems and Computer Science. The quality of this research is of course, only evaluated by the reader.
\section{Research Questions}
\large{Suggestions}
\begin{enumerate}
\item Hva gjør det vanskelig for en bruker å benytte seg av IKT som verktøy?
\item Hva er grunnen til at en bruker, i ett land med begrensede ressurser, ikke får utnyttet IKT verktøy maksimalt?
\item Hvilke hinder er det som står imellom bruker og IKT som verktøy i et land med begrensede ressurser og i en helse-setting?
\item Hva karakteriserer utfordringen, "å ta ibruk IKT-verktøy" i helse-sektoren i et land med begrensede ressurser?
\item I denne oppgaven, hvordan skal jeg vinkle målet med IT (Tar gjerne imot forslag)?
	\begin{enumerate}
	\item Få slutt på fattigdom?
	\item Øke livskvaliteten til folket?
	\item Mer kontroll til staten?
	\item Øke kunnskapsbasen om informasjons systemer?
	\item Ved bruk av IT, kan en bruke begrensede ressurser mer effektivt?
	\end{enumerate}
\end{enumerate}