\chapter{Introduction}

Free and open source initiatives can provide software that before had a price tag.
Technology is ever getting cheaper and internet penetration rates are ever increasing.
The technical requirements are being met. 
Therefore there are a puzzle to me why the technologies are not being taken advantage of.
It is well known that there should be a lot inefficiencies that could be remedied by introducing \gls{ict}'s in different areas.
In the more developed countries of the world we have been able to apply \gls{ict}'s to some level of success.
In the less developed countries of the world \gls{ict} projects have a tendency to fail or the progress are not moving forward as fast as expected.
The overall theme of this paper is therefore;\vspace{1cm}\\
\textbf{Why is it then, that \gls{ict} have a hard time being applied when the opportunities are already there?}\vspace{1cm}\\
If the technologies are available and ready to being taken to use. 
This thesis will try to give an answer to this question. 
\newpage
The answer will be provided with an action research study taken place in one of the countries that are currently trying to take advantage of the rapid technological advancements of \gls{ict}'s.
The \gls{ar} project are executed in collaboration with an organization called \gls{hisp}.
They are currently trying to implement a \gls{is} system called \gls{dhis2} in several developing countries.
This may provide insiders perspective on the current challenges we are facing in this area. 
The action research focus primary on introducing \gls{ict}'s in the health sector of developing countries where the need for improvement are most critical. 
With this in mind I will also try to give answers to the following research questions:
%Forsknings spørsmål
\begin{itemize}
\item \textbf{How can we then improve the process of introducing new technology in developing countries?}
\item \textbf{What are the necessary actions needed in order to increase the success rates of \gls{ict} initiatives?}
\item \textbf{Are there any new challenges that have not yet been highlighted as a result of this \gls{ar}-project?}
\end{itemize}
