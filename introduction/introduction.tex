\chapter{Introduction}


\section{Philosophy}
I will try to take on both an interpretive and a positivist view of my research.
The reason I do this is that I believe that one does not exists without the other.
No one really have a shared reality, and is never completely different.
The positive researcher will concentrate on the shared knowledge in a community, while the interpretive will try to harmonize the different realities. The users of the knowledge I am trying to create are the academics, focused in the field of Information Systems and Computer Science. The quality of this research is of course, only evaluated by the reader.

\section{Purpose}
My reason for doing this is divided. 
Firstly, I am a student using research to add to my own knowledge in the field of computers in order to being able to offer a better service in the computer industry.
Secondly, I am trying to add to the body of knowledge in the academic literature.
Starting this research project, I have privilege to know the organization HISP. 
Through HISP I've been participating in the configuration and implementation of a open source software called DHIS2 in Rwanda. After a quick analysis of the requirements I took notice that the software in question should indeed be able to offer solutions to each one of them. So why is it that it is not currently doing so? 
So this is my purpose, to find out why a software that to me seems to support all the necessary requirements is not doing so.

\section{Products of this research}
By participating in the configuration and implementation of DHIS2 as an intern at the MSH, one of the products of this research will be a working computer application. The other part is an in-depth study of this process. Hopefully contributing to the collection of data existing on the topic of ICT's in developing countries.


\section{Research Questions}
\large{Suggestions}
\begin{enumerate}
\item Hva gjør det vanskelig for en bruker å benytte seg av IKT som verktøy?
\item Hva er grunnen til at en bruker, i ett land med begrensede ressurser, ikke får utnyttet IKT verktøy maksimalt?
\item Hvilke hinder er det som står imellom bruker og IKT som verktøy i et land med begrensede ressurser og i en helse-setting?
\item Hva karakteriserer utfordringen, "å ta ibruk IKT-verktøy" i helse-sektoren i et land med begrensede ressurser?
\item I denne oppgaven, hvordan skal jeg vinkle målet med IT (Tar gjerne imot forslag)?
	\begin{enumerate}
	\item Få slutt på fattigdom?
	\item Øke livskvaliteten til folket?
	\item Mer kontroll til staten?
	\item Øke kunnskapsbasen om informasjons systemer?
	\item Ved bruk av IT, kan en bruke begrensede ressurser mer effektivt?
	\end{enumerate}
\end{enumerate}

\section{Background}

\section{Motivation}