\chapter{Method}
\section{Action Research}	
\begin{figure}
\centering
\caption{The Basic Action Research Avtivities}
\label{fig:basicactivity}
\end{figure}
Action research has it's origins in the \gls{usa} around the 1940--50's by a guy named Lewin.
He was applying social psychology techniques to practical social problems. 
Similar, but independently work was sone at the Tavistock Institute in \gls{uk} around 1950--60's.
At the institute they where trying to treat patients suffering from World War 2 experiences. 
They did not have a working theory of how they was going to heal the patients, but used a method that uses the basic principles of action research.
They planned, acted and reflected on their interventions with their patients. 
By going through this cycle they were able to form a knowledge base.
This knowledge base could then be applied when treating patients suffering from war trauma.
Since then, action research has been used by proffesionals to investigate and improve own practises.
In the most general terms, people are able to do research while doing their everyday jobs.
Action research is most aplicable in problem situations where it's not has easy to use academic models like abstract hypothesis, mathematical proofs or lab experiments. Like in social settings where qualitative data are most useful while trying to make sense of the messiness of human interactions.
In action research there is a focus on change and how the interventions play out in the real world.
The action researcher are therefore usually a member of the team working in the chosen problem situation.
This makes the researcher able to get an "inside" interpretation of the context of the problem.
The action research cycle are often extended to include some extra activities. 
\begin{table}
\begin{tabular}{m{2cm} m{2cm} m{8cm}}
\hline
\multirow{2}{*}{Plan} & Diagnose & Identifying the nature of the problem situation, including all interrelated factors and develop a working theory about the situation and how it maight be changed.\\
\cline{2-3}
 & Plan & Specifying actions that should allevieate the situation. \\
\hline
Act & Act & Taking action in the agreed area of application in line with the plan.\\
\hline
\multirow{2}{*}{Reflect} & Evaluate & Establishing whether the theoretical effects of the action were realized and whether they did indeed relieve the problem. \\
\cline{2-3}
 & Reflect &  Deciding what has been achieved in terms of both practical outcomes and new knowledge. At this point there should also be decided if there is a need for another cycle. \\
\hline
\end{tabular} 
\caption{Five stage Action Research\cite{bjo:risc}}
\label{tab:5stagear}
\end{table}
A action research project might be difficult to plan to the detail due to the nature of the problem situation. 
Action research takes place in a real life scenario, and are more difficult to control compared to for an example a lab.
The planning of the intervention must therefore be highly adaptable.





\begin{quotation}
There is a conception that encourages a separation of theory from practice because published research is read more by producers of research than by practitioners. As a result, practitioners and their clients complain more and more frequently about the lack of relevance of published research for the problems they face and about the lack of responsiveness of researchers to meeting their needs.
Action research is an answer to this problem. 


The term was introduced by Kurt Lewin in 1946. The process is conceibed as a spiral of steps, each of which is composed of a circle of planning, action and fact-finding about the result of the action.

Action research facilitates the development of techniques which are called parctics. 
The practics would provide the action researcher with know-how such as how to create setting for organizational learning, how to act in unprescribed nonprogrammed situations, how to generate organizational self-help, how to establish action guides where non exist, how to review, revise, redifine the system of which we are part, how to formulate fruitful metaphors, constructs, and images for articulating a more desirable future. 
Action research is a procedure for generating knowledge for understanding and managing the affairs of organizations.
\subsection{Pitfalls}
Post-rationalizing and retro-interpretations.
\\
\end{quotation}
\cite{car:rmn}
\cite{bjo:risc}
\cite{assess:susman}

\section{Research Analysis}
This section will try to give an abstract overview of the research process in terms of activities drawn out from the content. 
A more descriptive representation follows in the chapter \ref{ch:case}.  
\subsection{Diagnosis}
The diagnosis activity in the action research are used to identify the nature of the problem situation. 
The main objective being to obtain a working theory of the context.
In this research project it involved a series of steps.
For starters it involved a preparing literature review that was supposed to work as a theoretical foundation from the academic community.
Topics of interest were "ict4d", "ICT in developing countries", "E-health", "Action Research" and "transistion strategy".
With this in mind we moved over to an introduction that was more case specific.
This was provided by the \gls{hisp} team in Oslo. One of the organization's involved in the problem situation.
Here we discussed topics more technical and practical. 
Some problems that had common solutions to them were presented and a introduction to common technologies were introduced.
At the same time important introductions were made so that all areas of expertise were covered before leaving from Norway to were the actual case took place, Rwanda.
As a last part of the diagnosis and most significant to the problem situation were the introduction to collaborating partners of the \gls{hmis} team were I would work as member of the team while performing the action research.
For starters we had meetings and briefings about the status. Documents where exchanged, introductions made and presentations where held in order to briefly get into the workings of the situation problem.
In parallell with this, continuous email communication with representatives from the universities involved, the organization \gls{hisp} and the \gls{moh}. This in particular helped to smoothen out the misinterpretations that was made along the way. 
As a result of the diagnose a set of objectives were defined, elaborated in chapter \ref{ch:case}.
\subsection{Planning}
While planning for the appropriate interventions we had to adjust to a plan already made by the \gls{hmis} team.
Our interventions had to be aligned with the activities already agreed upon between the \gls{hmis} and the \gls{chd}.
In order to achieve this we identified the parts of the plan that related to the objectives that resulted from the diagnosis.
By combining the two we made it possible to combine the interests of the researcher and the organization. 
In order to facilitate the research process we made some adjustments to the setting of interventions. 
Due to critical time contraints we chose to set up a test environment to actualize the interventions rather than testing in a real scenario.
Also this would make it more easy to test and make demonstrations for the clients along the way.

\subsection{Intervention}
The interventions was based on meeting the objectives resulted from the diagnosis. 
Actions taken were partly configuring the system already developed and partly developing new solutions.
Actions concerning configurations were made by using the existing knowledge of the organization.
The system was quite familiar to all of those involved.
By combining the knowledge of the team and continually testing in the test-environment we were able to make evidence of our knowledge.
Since some of the objectives were not supported fully, we decided to make an improvised development cycle along with the configurations. 
Developing supporting applications that would work as an integrated part of the system became a necessety when the objectives no longer were supported by the existing system.
\subsection{Evaluation \& Reflection}
Optimaly we would have the users of the system evaluate it. 
Due to time constraints this was not possible for this research project. 
The evaluation of the project was done by presenting the solutions in the test environment in three parts.
Firstly to the team members. Then to the clients and finally to the clients supervisor and advisors.
The reflection part of the research cycle will come through with this paper as lessons learned.
Another research cycle was unfortunately not an option due to time constraints.
It should be mentioned that the research looses half of it's learning potential since the cycle could not be completed within the timeframe set at the problem situation. 
Making the knowledge created by this research project only contributing to the academics and not the organization that hosted the project, which are in conflict with some of the problems that action research is supposed to remedy.

\section{Evaluation of Research}
Action research very loose boundries. This as caused some to argue that this type of research needs more rigor. 
In order to achieve this there are some helpful principles that can guide the aspiring action researcher. 
The prinsiples facilitate the type of action research now known as \gls{car}.
This is the form of \gls{ar} that is the most traditional. 
It generally has the structure outlined in table \ref{tab:5stagear}, with 5 steps in the \gls{ar} cycle.
In a orderly fashion we go through them here in order to get critical evaluation of the \gls{ar}-project. 
\subsection{Researcher Client Agreement}
\begin{table}
\centering
\begin{tabular}{p{2cm} p{8cm}}
\hline
1a & Did both the researcher and the client agree that \gls{car} was the appropriate approach for the organizational situation? \\
1b & Was the focus of the research project specified clearly and explicity? \\
1c & Did the client make an explicit commitment to the project? \\
1d & Were the roles and responsibilities of the researcher and client organization members specified explicitly? \\
1e & Were project objectives and evaluation measures specified explicitly? \\
1f & Were the data collection and analysis methods specified explicitly? \\
\hline
\end{tabular}
\caption{Evaluating \gls{rca}}
\label{tab:evarca}
\end{table}
The agreement between the client (\gls{moh}, gls{hisp}) and the researcher was that the researcher should contribute to the organization as an intern.
This put the researcher role in the organization most suitable. 
As an intern the researcher would work on the configuration and development of the \gls{clmis} which was appropriate concerning the research approach. The project lacked specification prior to the start. This made it difficult to plan for the different preperations necessary.
The client already had a big commitment to the project. 
Clearly a national system that would manage health supplies for the \gls{chw}'s is something that are loosely commited to. 
Both political and technical initiatives was a driving force for progress. 
After the diagnosis we had pretty clear objectives that could be described as use cases.
A drawback that comes from not being able to specify the focus of the project was the inability to prepare for more data generation methods.
The project would tended to being focused on to much intervention which took time from the data generation and reflection part.


\subsection{Cyclical Process Model}
\begin{table}
\centering 
\begin{tabular}{p{2cm} p{8cm}}
\hline
2a & Did the project follow the \gls{cpm} or justify any deviation from it? \\
2b & Did the researcher conduct an intependent diagnosis of the organizational situation? \\
2c & Were the planned actions based explicitly on the result of the diagnosis? \\
2d & Were the planned actions implemented and evaluated? \\
2e & Did the researcher reflect on the outcomes of the interventions? \\
2f & Was this reflection followed by an explicit decision on whether or not to proceed through an additional process cycle? \\
2g & Were both the exit of the researcher and the conclusion of the project due to either the project objectives being met or some other claerly articulated justification? \\
\hline
\end{tabular}
\caption{Evaluating the \gls{cpm}}
\label{tab:evacpm}
\end{table}
The project did follow the \gls{ar} cycle in the beginning, but after had some deviations after the interventions started.
The size of the interventions took up to much of the timeframe. 
It was then necessary to partly evaluate and act at the same time. Three demo's were conducted in order to evaluate our progress. 
The reflection and lessons learned phase were conducted outside the organization, making another research cycle out of the question.
The time frame of the project planned was therefore to narrow for the planned interventions. 

\subsection{Theory}
\begin{table}
\centering
\begin{tabular}{p{2cm} p{8cm}}
\hline
3a & Were the project activitist guided by a theory or set of theories? \\
3b & was the domain of investigation and the specific problem setting relevant and significant to the interests of the researchers community of peers as well as the client? \\
3c & Was a theoretically based model used to derive the causes of the observed problem? \\
3d & Did the planned intervention follow from this theoretically based model? \\
3e & was the guiding theory or any other theory used to evaluate the outcomes of the intervention? \\
\hline
\end{tabular}
\caption{Evaluating the Theory}
\label{tab:evath}
\end{table}
As part of the diagnosis there was conducted a literature review that guided the research.
There has been some complaints of little research data coming from developing countries.
An \gls{ar} project would therefore contribute to the qualitative database and should be of interest to the community.
Theory have suggested that a lack of resources are a common problem in these situations and was also found to be true.
Although resources here should be seen as technical expertise.


\subsection{Change through Action}
\begin{table}
\centering
\begin{tabular}{p{2cm} p{8cm}}
\hline
4a & Were both the researcher and client motivated to improve the situation? \\
4b & Were the problem and its hypothesized causes specified as a result of the diagnosis? \\
4c & Were the planned actions designed to address the hypothesized causes? \\
4d & Did the client approve the planned actions before they were implemented? \\
4e & Was the organization situation assessed comprehensively both before and after the intervention? \\
4f & Were the timing and nature of the actions taken clearly and completely documented? \\
\hline
\end{tabular}
\caption{Evaluating Cange through Aciton}
\label{tab:evachange}
\end{table}
Both the client and the researcher was indeed motivated to improve the situation.
The client as the initiater of the project and the researcher after investing much time found that the project was indeed worth while.
The diagnisis of the project lead to clear objectives that easily could be translated to appropriate interventions.
Since researcher and the client was "under the same roof" the interventions had to be agreed upon. 
Assessing the organizatin was done through the diagnosis. Unfortunately the lasting effects of the interventions could not be assessed. 
A report of the actions taken was produces by the researcher and delivered to the organization. 
Like wise this paper will serve as documentation to the academics. 
\subsection{Learning through reflection}
\begin{table}
\centering
\begin{tabular}{p{2cm} p{8cm}}
\hline
5a & Did the researcher provide progress reports to the client and organization members? \\
5b & Did both the researcher and the client reflect upon the outcomes of the project? \\
5c & Were the research activities and outcomes reported clearly and completely? \\
5d & Were the results considered in terms of implications for further action in this situation?  \\
5e & Were the results considered in terms of implications for action to be taken in related research domains? \\
5f & Were the results considered in terms of implications for the research community? \\
5g & Were the results considered in terms of the general applicability of \gls{car}?  \\
\hline
\end{tabular}
\caption{Evaluating Learning through Reflection}
\label{tab:eva}
\end{table}
A progress report was made at the end of interventions. This was on the form of a trip report.
As stated earlier, the reflection did not accour with the client and the research did not benefit from assessing the outcomes of the interventions. Through this paper and the trip report the acitvities would be documented properly.

\cite{car:rmn}
