\chapter{Method}
\section{Action Research}	
\begin{figure}
\centering
\caption{The Basic Action Research Avtivities}
\label{fig:basicactivity}
\end{figure}
Action research has it's origins in the \gls{usa} around the 1940--50's by a guy named Lewin.
He was applying social psychology techniques to practical social problems. 
Similar, but independently work was sone at the Tavistock Institute in \gls{uk} around 1950--60's.
At the institute they where trying to treat patients suffering from World War 2 experiences. 
They did not have a working theory of how they was going to heal the patients, but used a method that uses the basic principles of action research.
They planned, acted and reflected on their interventions with their patients. 
By going through this cycle they were able to form a knowledge base.
This knowledge base could then be applied when treating patients suffering from war trauma.
Since then, action research has been used by proffesionals to investigate and improve own practises.
In the most general terms, people are able to do research while doing their everyday jobs.
Action research is most aplicable in problem situations where it's not has easy to use academic models like abstract hypothesis, mathematical proofs or lab experiments. Like in social settings where qualitative data are most useful while trying to make sense of the messiness of human interactions.
In action research there is a focus on change and how the interventions play out in the real world.
The action researcher are therefore usually a member of the team working in the chosen problem situation.
This makes the researcher able to get an "inside" interpretation of the context of the problem.
The action research cycle are often extended to include some extra activities. 
\begin{table}
\begin{tabular}{m{2cm} m{2cm} m{8cm}}
\hline
\multirow{2}{*}{Plan} & Diagnose & Identifying the nature of the problem situation, including all interrelated factors and develop a working theory about the situation and how it maight be changed.\\
\cline{2-3}
 & Plan & Specifying actions that should allevieate the situation. \\
\hline
Act & Act & Taking action in the agreed area of application in line with the plan.\\
\hline
\multirow{2}{*}{Reflect} & Evaluate & Establishing whether the theoretical effects of the action were realized and whether they did indeed relieve the problem. \\
\cline{2-3}
 & Reflect &  Deciding what has been achieved in terms of both practical outcomes and new knowledge. At this point there should also be decided if there is a need for another cycle. \\
\hline
\end{tabular} 
\caption{Five stage Action Research\cite{bjo:risc}}
\end{table}
A action research project might be difficult to plan to the detail due to the nature of the problem situation. 
Action research takes place in a real life scenario, and are more difficult to control compared to for an example a lab.
The planning of the intervention must therefore be highly adaptable.




\subsection{The Research-Client Agreement}
\begin{table}
\centering
\begin{tabular}{p{2cm} p{6cm}}
\hline
\hline
1 & Did both the researcher and the client agree that \gls{car} was the appropriate approach for the organizational situation? \\\hline
2 & Was the focus of the research project specified clearly and explicity? \\\hline
3 & Did the client make an explicit commitment to the project? \\\hline
4 & Were the roles and responsibilities of the researcher adn client organization members specified explicitly? \\\hline
5 & Were project objectives and evaluation measures specified explicitly? \\\hline
6 & Were the data collection and analysis methods specified explicitly? \\\hline
\hline
\end{tabular}
\caption{Criteria for the \gls{rca}}
\label{tab:rcacriteria}
\end{table}

Table \ref{tab:rcacriteria} lists some criterias one should consider before starting an action research project.
The \gls{rca} guiding foundation of the research project. In order for the \gls{rca} to be effective it is necessary for the client to be aware of how \gls{car} works. There are both advantages and drawbacks that should be considered before embarking on an action research project. 
It is very much a joint venture. 
The agreement should contain mutual guarantees for behaviour in the problem situation. 
The agreement helps to facilitate the relationship between the researcher and the client by having the client participate when determining the researchers goals, implemtations and the assessments of the outcomes.
It is important that there is an agreement of the focus of the project as organizational constraints may conflict with research justifiable interventions. This also includes that the timeframe of the research project may differ in that of an organizational timeframe. 
Further there is called for a commmitment from the client to the research project. In order to evade confilict there should be an initial set of roles before the project starts. Since the researcher are serving two masters, the academic society and the organization there must be an agreement of measures. Those of the organization may not be valid in the research community and vice verca.
Classified information may be of use to the researcher, but may not be beneficial for the organization to share. Obviously there should be an agreement of what documents and information that should be available to the researcher.

\begin{quotation}
There is a conception that encourages a separation of theory from practice because published research is read more by producers of research than by practitioners. As a result, practitioners and their clients complain more and more frequently about the lack of relevance of published research for the problems they face and about the lack of responsiveness of researchers to meeting their needs.
Action research is an answer to this problem. 


The term was introduced by Kurt Lewin in 1946. The process is conceibed as a spiral of steps, each of which is composed of a circle of planning, action and fact-finding about the result of the action.

Action research facilitates the development of techniques which are called parctics. 
The practics would provide the action researcher with know-how such as how to create setting for organizational learning, how to act in unprescribed nonprogrammed situations, how to generate organizational self-help, how to establish action guides where non exist, how to review, revise, redifine the system of which we are part, how to formulate fruitful metaphors, constructs, and images for articulating a more desirable future. 
Action research is a procedure for generating knowledge for understanding and managing the affairs of organizations.
\subsection{Pitfalls}
Post-rationalizing and retro-interpretations.
\\
\end{quotation}
\cite{car:rmn}
\cite{bjo:risc}
\cite{assess:susman}

\section{Research Analysis}
This section will try to give an abstract overview of the research process in terms of activities drawn out from the content. 
A more descriptive representation follows in the chapter \ref{ch:case}.  
\subsection{Diagnosis}
The diagnosis activity in the action research are used to identify the nature of the problem situation. 
The main objective being to obtain a working theory of the context.
In this research project it involved a series of steps.
For starters it involved a preparing literature review that was supposed to work as a theoretical foundation from the academic community.
Topics of interest were "ict4d", "ICT in developing countries", "E-health", "Action Research" and "transistion strategy".
With this in mind we moved over to an introduction that was more case specific.
This was provided by the \gls{hisp} team in Oslo. One of the organization's involved in the problem situation.
Here we discussed topics more technical and practical. 
Some problems that had common solutions to them were presented and a introduction to common technologies were introduced.
At the same time important introductions were made so that all areas of expertise were covered before leaving from Norway to were the actual case took place, Rwanda.
As a last part of the diagnosis and most significant to the problem situation were the introduction to collaborating partners of the \gls{hmis} team were I would work as member of the team while performing the action research.
For starters we had meetings and briefings about the status. Documents where exchanged, introductions made and presentations where held in order to briefly get into the workings of the situation problem.
In parallell with this, continuous email communication with representatives from the universities involved, the organization \gls{hisp} and the \gls{moh}. This in particular helped to smoothen out the misinterpretations that was made along the way. 
As a result of the diagnose a set of objectives were defined, elaborated in chapter \ref{ch:case}.
\subsection{Planning}
While planning for the appropriate interventions we had to adjust to a plan already made by the \gls{hmis} team.
Our interventions had to be aligned with the activities already agreed upon between the \gls{hmis} and the \gls{chd}.
In order to achieve this we identified the parts of the plan that related to the objectives that resulted from the diagnosis.
By combining the two we made it possible to combine the interests of the researcher and the organization. 
In order to facilitate the research process we made some adjustments to the setting of interventions. 
Due to critical time contraints we chose to set up a test environment to actualize the interventions rather than testing in a real scenario.
Also this would make it more easy to test and make demonstrations for the clients along the way.

\subsection{Intervention}
The interventions was based on meeting the objectives resulted from the diagnosis. 
Actions taken were partly configuring the system already developed and partly developing new solutions.
Actions concerning configurations were made by using the existing knowledge of the organization.
The system was quite familiar to all of those involved.
By combining the knowledge of the team and continually testing in the test-environment we were able to make evidence of our knowledge.
Since some of the objectives were not supported fully, we decided to make an improvised development cycle along with the configurations. 
Developing supporting applications that would work as an integrated part of the system became a necessety when the objectives no longer were supported by the existing system.
\subsection{Evaluation \& Reflection}
Optimaly we would have the users of the system evaluate it. 
Due to time constraints this was not possible for this research project. 
The evaluation of the project was done by presenting the solutions in the test environment in three parts.
Firstly to the team members. Then to the clients and finally to the clients supervisor and advisors.
The reflection part of the research cycle will come through with this paper as lessons learned.
Another research cycle was unfortunately not an option due to time constraints.
It should be mentioned that the research looses half of it's learning potential since the cycle could not be completed within the timeframe set at the problem situation. 
Making the knowledge created by this research project only contributing to the academics and not the organization that hosted the project, which are in conflict with some of the problems that action research is supposed to remedy.
\section{Method Summary}

