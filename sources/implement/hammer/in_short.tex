\documentclass{article}
\usepackage[utf8]{inputenc}
\title{{\Huge Summary of: \\ ``Reengineering Work''} \\ {\small by Michael Hammer}}
\author{Kenneth Børtveit}
\begin{document}
\maketitle
Tries to do business the same way, but uses technology to make it automated. Nothing new just the same but machines doing the labour.
This could work for speeding up processes, but issues in the process still remains.'

The old way was characterized by efficiency and control. The new way is innovation, speed, service and quality. I'll just repeat that:
\begin{itemize}
\item Innovation
\item speed
\item service
\item quality
\end{itemize}

The author thinks we should throw out the old system and just reinvent everything. I think this could be a good idea, but one has to remember that there is some quality in that which has existed and prevailed for many, many decades. It is the foundation from which innovation can exist. Unless we have a strong foundation, there isnt the possibility of even trying. I don't think that many people agree with what Dubai is doing. Making all new shit, pushing everything to the limits of what mankind can do, but when the money stops, it just do that. No more innovation, no more new tricks. So, if the old way was automated, and we had a foundation of continuing stream of money and this was automated. It would in fact contribute to the list above. 

The author proposes that it takes some courage to change, but then again, if you have the foundation there would not even be a problem, cause you can take the hit. And once the organization is comfortable in the new setting just slowly migrate to that setting. 

\section{Ford and MLB}
Ford was planning to cut costs by making accounting take less money. Their goal was to go from 400 to 320, cutting it by 20\%. When looking at Mazda they observed that they managed with only 5 people. This got them thinking and decided that they should set a better goal. Instead they managed to cut by 75\%. Amazing you say? It's just looking that others do better and then realizing that one can do the same. If they studied Mazda some more they would probably find some other things that they could improve. Just look at the world today. After realizing that everybody can do the same, they just copy it and do it for them selves. Industry countries has probably set the standard globaly and other countries are beginning to pick up the pace, but with a little twist. They to has benefits from their culture that they integrate in their own way and thereby synthezing making it to something better. It isnt groundbreaking, it's just the way it is and always will be. The lessons and competance a country hasn't dissapeared by introducing Industry. That is in turn integrated and synthesized with their own way of doing things. Of course they will use the parts that benefits and throw away the things that isn't working. This will of course make something new, but it is not groundbreaking and a complete revolution, these things happen just the way it does every time. Like the way we learned thinking from the greeks, math from the arabs and industry from somewhere else. Everything comes together and we keep what we want and need and discard the rest.

MLB had applications that could take several weeks to complete. Suddenly they decided that this nonesense had to stop. So they decided to be 60\% more productive. Take note of the decision here. They have a clear goal and made the decision to do so. Then they laid out some plans that used technology to achieve it, with greater efficiency in mind. They reduces their staff to by the way. They followed through and behold! they could do an application in 2 -- 5 days. One can get caught up in that this is a new way of thinking, but the fact is that making plans, take decision and setting goals is nothing new. 

\section{The essance of reengineering}
The author basicly says that reengineering is changing and making up new rules and take the fundamental assumtions and challenge them. That is all good. Why not? Do we need to go to church every sunday, is this fundamentally important for our well being in these times? The rules will change after people understand them, get a better perspective and can play with them a little and once understood, they are not rules anymore, but guidelines for making business. Rules in general makes things slow and unadaptable. Things will never stay the same, therefore one has to adapt to the new situation continually. This is not something new either. We do it all the time. It's called by some people evolution. The speed might scare some people, but the old ones are allways scared of change anyway. If one thinks that this is new, think again. Why does the old get pushed out of society I wonder?


And that is all we have time for today.


\end{document}