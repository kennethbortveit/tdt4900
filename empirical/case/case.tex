\chapter{Case}
Jeg funnet ut at brukere av systemet kan ta feil av 1 og l, og 0(null) og O. Dette er jo selvfølgelig viktig å tenke på når vi skal lage koder som skal brukes til å rapportere.
Dette blir jo en del av diagnosen. Jeg lærte det av Lars Roland og fikk det litt bekreftet av Randy.

Videre er det en del basis kunnskap som burde vært på plass her som ikke vi lærer på skolen heller. Som kommandoer i Linux. Jeg er klar over at dette ikke har mye fokus blant forskere, men ting går tregt på grunn av det tror jeg.
Forsknings koblingen vil kanskje gå på å få brukere til å stole på at datamaskinen gjør det den skal når en skal utføre en operasjon.

Har masse å skrive om virker det som.

\section{Stakeholders}
\subsection{NTNU}
My university.                                                                                                                                       
\subsection{HISP/UiO}
Organization responsible for making the DHIS2.
\subsection{MSH}
Company that would like to help to build better health information systems.
\subsection{HMIS}
The team that works at the ministry of health.
\subsection{Community Help Desk}
These are the clients.

\section{SMS Reporting Service}
\subsection{DHIS2}
\subsubsection{SMS Gateway}
\subsubsection{DHIS Analysis}

\section{Design and Creation of the User Importer}