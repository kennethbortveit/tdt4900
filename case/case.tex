\chapter{Case}
\section{Background}
There has been some interest in the area of \gls{sms} reporting from the \gls{uio}.
The \gls{dhis2} software supporting this functionality has been developed, but not yet been used.
The \gls{hmis} team at the \gls{moh} in Rwanda has for some time been wanting to use \gls{dhis2} in order to make a system for keeping track of \gls{chw}'s essential drugs and supplies. The system, \gls{clmis}, should be able to track \gls{chw}'s stock and distributions of these items. 
The \gls{hmis} team are actually working for the \gls{chd} who are the clients in this case. 

The current system is primarely a pull system where \gls{chw}'s make monthly visits to their local \gls{hc} \gls{chw} supervisors in order to resupply. 
\begin{quotation}
In order for these \gls{chw}'s to provide uninterupted care to their communities, it is essential to have access to the essential drugs and supplies these health workers dispense.
\end{quotation}

Rwanda is now in the process of rolling out a national \gls{elmis} that is supposed to cover all levels of the health system, but this does not include the $\approx 45000$\gls{chw}'s in $\approx 15000$villages.
This is were the \gls{clmis} comes in. 
With \gls{dhis2} as a base software \gls{chw}'s will be able to report data on what they receive and has in stock of the essential drugs and supplies. 
Further, the plan is to integrate \gls{clmis} with the national \gls{elmis} in order to have interoperability between systems. 

\section{Networks of Action}
As mentioned by Eric, Jørn and Sundeep, on key to make this possible is the network of action. 
As a student-researcher I've been able to get a position as an intern at \gls{msh}.  
The core of this initiative is the \gls{chd}. They have asked \gls{hmis} for support on developing \gls{clmis}. 
The \gls{hmis} team has support from both \gls{msh} and \gls{hisp}. 
\subsection{Description of the different participants here}

\section{Objectives}
In order to make the case managable for a research project it was limited to four objectives. 

\begin{description}
\item[\#1:] Send SMS and email notifications based on rules.
\item[\#2:] Send SMS and email reminder if a report is more than 4 days delayed.
\item[\#3:] If user data does not map correctly user feedback should be provided.
\item[\#4:] A functional SMS based reporting system.
\end{description}

These objectives are somewhat simplified in order to be easier to work with. A more elaborate description folllows. 
\subsection{Objective \#1}
Notifications here are meant as in the broadest of meanings. The idea is that the system should be able to communicate with the \gls{chw}'s based on some configuration. In this case, a notification could mean a resupply order or an alert. Rules would then be related to thresholds or algorithms. For an example, resupply order would be generated by an algorithm that calculates how much of each supply item the \gls{chw} needs. 
\subsection{Objective \#2}
This objective is straight forward. If a \gls{chw} in charge of reporting at a village does not report after 4 days of the previous reporting month, a reminder should be sent. 
\subsection{Objective \#3}
Sometimes when a \gls{chw} reports data, syntax error may happen. It is also preferable to have some kind of feedback when everything is just fine. Just to know that everything is working. 
The appropriate instructions for fixing mistakes should also be in the feedback from the system.
\subsection{Objective \#4}
In this case a functional reporting system would be a system that is ready to receive \gls{sms} reports from the \gls{chw}'s. These messages are stored in the \gls{clmis} database ready to be analyzed. 

\section{Refining and Defining the Requirements}
As a part of a diagnosis we started out with trying to define usecases for each of the objectives. 
This would make it more clear what needed to be done in order to meet them.
It was very diffiult to pinpoint exactly what needed to be done bacause of the projects size. 
\gls{hmis} was in charge of configuring and develop the system. \gls{hmis} was doing this for the \gls{chd}, both located in the the same department, \gls{moh}. 
Collecting the requirements would then be based on what we understood from what the \gls{chd} could tell us. \gls{hmis} had already made some progress on this part.  

\begin{figure}
\centering
\includegraphics[width=\textwidth]{case/img/chwSupplyChainFuture}
\caption{\gls{chw} Supply Chain in the Future}
\label{chwSupplyChainPresent}
\end{figure}

Figure \ref{chwSupplyChainFuture} shows the desired \gls{bpm}. The specifics did not allways match what had previously been discussed, but the important part was to get an overall picture of how thing should work. For an example we will see that the \gls{chw}'s would rather report on what they receive instead of what they dispense. After analyzing the \gls{chw} supply chain \gls{bpm} we found the following. 
Activity 1, 2, 3, 7 was supported as long as the \gls{chw} had a mobile phone. After discussing it with one of \gls{chd}'s team members, it was fairly safe to assume this. 
Activity 4 relates directly to objective \#4.
Activity 6 relates to objective \#3. 
Activity 5 relates to objective \#2 and Activity 8 and 11 to objective \#1. Activity 9--10, 12--17 should supported as long as the objectives were met. 
This puts our objectives in context of a bigger picture.

\subsection{Use Cases}
\begin{table}
	\centering
	\begin{tabular}{|p{5cm}|p{7cm}|}
		\hline
		\multicolumn{2}{|c|}{\textbf{Send SMS and Email Notifications}}\\
		\hline
		\textbf{Goal:} & Create orders\\
		\hline
		\textbf{Primary Actor:} & System\\
		\hline
		\multirow{3}{*}{\textbf{Secondary Actor:}}	& Cell CHW Supervisor \\
																								& HC CHW Supervisor \\ 
																								& District Pharmacist \\
		\hline
		\multirow{4}{*}{\textbf{Main Success Scenario:}}	& 1. CHW reports distributed and stock values. \\
																											& 2. System processes report. \\
																											& 3. System calculates essential drugs needed for each level. \\
																											& 4. System sends orders to cell, sector and district. \\
		\hline
		\textbf{Extensions:} & \\
		\hline
	\end{tabular}
	\caption{Textual Use Case: Send SMS and Email Notifications}
	\label{tab:notifications}
\end{table}


\begin{table}
	\centering
	\begin{tabular}{|p{5cm}|p{7cm}|}
		\hline
		\multicolumn{2}{|c|}{\textbf{Send SMS and Email Reminders}}\\
		\hline
		\textbf{Goal:} & Send reminder \\
		\hline
		\textbf{Primary Actor:} & System \\
		\hline
		\multirow{2}{*}{\textbf{Secondary Actor:}}	& CHW \\
																								& Cell CHW Supervisor \\
		\hline
		\multirow{5}{*}{\textbf{Main Success Scenario:}}	& 1. CHW misses report deadline. \\
																											& 2. 5 days goes by. \\
																											& 3. System sends reminder by email and SMS. \\
																											& 4. Another 5 days goes by. \\
																											& 5. System sends reminder by email and SMS. \\
																											
		\hline
		\textbf{Extensions:} & \\
		\hline
	\end{tabular}
	\caption{Textual Use Case: Send SMS and Email Reminders}
	\label{tab:reminders}
\end{table}


\begin{table}
	\centering
	\begin{tabular}{|p{5cm}|p{7cm}|}
		\hline
		\multicolumn{2}{|c|}{\textbf{Send Report Feedback}} \\
		\hline
		\textbf{Goal:} & Process SMS message\\
		\hline
		\textbf{Primary Actor:} & System \\
		\hline
		\textbf{Secondary Actor:} & Community Health Worker \\
		\hline
		\multirow{6}{*}{\textbf{Main Success Scenario:}}	& 1. CHW reports data incorrectly by SMS. \\
																											& 2. System receives SMS. \\
																											& 3. SMS triggers feedback message. \\
																											& 4. CHW corrects message and re-sends report. \\
																											& 5. System processes SMS. \\
																											& 6. System updates database. \\
		\hline
		\textbf{Extensions:} & \\
		\hline
	\end{tabular}
	\caption{Textual Use Case: Send Report Feedback}
	\label{tab:feedback}
\end{table}

\begin{table}
	\centering
	\begin{tabular}{|p{5cm}|p{7cm}|}
		\hline
		\multicolumn{2}{|c|}{\textbf{Report Using SMS}}\\
		\hline
		\textbf{Goal:} & Update Database \\
		\hline
		\textbf{Primary Actor:} & Community Health Worker\\
		\hline
		\textbf{Secondary Actor:} & System \\
		\hline
		\multirow{5}{*}{\textbf{Main Success Scenario:}}	& 1. CHW reports stock and distributed values of essential drugs. \\
																											& 2. System receives SMS. \\
																											& 3. System processes SMS. \\
																											& 4. System updates database. \\
																											& 5. System sends confirmation SMS to CHW. \\
		\hline
		\textbf{Extensions:} & \\
		\hline
	\end{tabular}
	\caption{Textual Use Case: Report Using SMS}
	\label{tab:smsreport}
\end{table}

As a seen in use case tables \ref{tab:notifications}, \ref{tab:reminders}, \ref{tab:feedback} and \ref{tab:smsreport}, the specifics did change, along with the development process, but it gave us the necessary guidelines to understand the desired outcome.
The obstacles then became somewhat clearer. The \gls{chw}'s needed a server to communicate with and the server needed to be able to communicate with the \gls{chw} supervisors at the different levels in the health hierarchy.
The communication channels that should be used between the system and the users would be email and \gls{sms}. Email support are possible to set-up without involving any other parties, but \gls{sms} on the other hand are somewhat tricker. Here we have to include a mobile company in order to proparly test the service. This service also includes using software and hardware outside of the department.


\section{Planning}
\begin{figure}
\centering
\includegraphics[width=\textwidth]{case/img/lmisWorkPlan}
\caption{Activity Plan For the \gls{chw} \gls{lmis}}
\label{fig:activityPlan}
\end{figure}
With the objectives put in context we could start planning the specific activites for intervention. 
In our case the \gls{hmis} team made the overall plan for the project as in figure \ref{fig:activityPlan}.
The objectives then relates to the following points of intervention, but taken into account that there are dependencies along the different activities.  
\begin{description}
\item[\textbf{Objective \#1}]\hfill
	\begin{description}
	\item[3.4]Develop algorithm for estimating resupply amounts.
	\item[3.7]Design SMS alerts for stocklow warnings to Cell and HC CHW coordinators.
	\item[3.8]Add parameters table for setting minstock, reorderlevel, defaultsupply by drug.
	\item[3.9]Design triggers to email reports to HC CHW supervisors and District Pharmacy staff.
	\end{description}
\item[\textbf{Objective \#2}]\hfill
	\begin{description}
	\item[3.5]Design SMS alerts for late stock reports.
	\end{description}
\item[\textbf{Objective \#3}]\hfill
	\begin{description}
	\item[3.6]Translate SMS feedback messages into Kinyarwanda.
	\end{description}
\item[\textbf{Objective \#4}]\hfill 
	\begin{description}
	\item[3.1]Import cell and village hierarchy into the DHIS-2.
	\item[3.2]Clean up and import all CHWs with phone numbers into DHIS-2 as users.
	\item[3.3]Create data elements for reporting (on the job training).
	\end{description}

\end{description}

The \gls{clmis} will in its final state run on servers at the \gls{ndc}.
This would then involve another party when trying to configure and develop the \gls{clmis}. 
Often taken for granted is stable power supply and internet access. 
In our case, this was not the case. On could experience power cuts on a daily basis.
And working directly on a server under these curcumstances is not very productive. 
Taking this into account we decided to set up a test environment that we could work with. 
Making our configurations and testing possible instantly before we make the changes on the live server at the \gls{ndc}. 
This duplicated our work some, but makes it easier to develop and configure. For an example, one does not need to stop everybody's work if one happens to play with the database to much. Also it makes it easier to divide tasks so that they can run in parallell. 

\section{Notes}
Based on the four objectives we made four use cases that was supposed to represent each one. Objective \#1 would be represented with use case \ref{tab:notifications}, objective \#2 with \ref{tab:reminders}, \#3 with \ref{tab:feedback} and \#4 with \ref{tab:smsreport}. These use cases worked as guidelines for our further work. They were not updated later on, because of the continuous updated requirements from \gls{chd} and other co-workers. That is one of the key characterizations of this project. The requirements kept on updating as more and more people got involved in the process. The more progress we made the less progress we made. As the project was coming more and more realized, more interest were made to the project, and more requirements were added. There were a kind of common understanding in the team. Once you got the picture, you didn't need to operate on a model anymore. Everyone kinda knew what needed to be done. The result of the diagnosis were essentially a clarification of what we were supposed to do and who are involved. The clients are \gls{chd}. A meeting took place and a list of contact information was exchanged. The users of the system are \gls{chw}'s, Cell \gls{chw} Supervisors, \gls{hc} \gls{chw} Supervisors and District Pharmacists. The basic idea is that the \gls{chd} would like to have \gls{hmis} make a system that enables \gls{chw}'s to report using \gls{sms} and based on this have automatic generated orders sent to the \gls{hc}'s and District Pharmacists.


Our planning phase became somewhat glued together with the intervention. And continually altered. New problems were made visible by the interventions we made, and took us back to the planning phase. Making it very difficult to follow the action research model. In a perfect world, it is possible to plan everything to the point, but in our case new knowledge about the system was discovered along with our interventions and in turn, our plans had to be changed. 

\begin{figure}
\centering
\includegraphics[width=\textwidth]{case/img/orderFlow}
\caption{Flow of Orders}
\label{fig:chworder}
\end{figure}

\begin{figure}
\centering

\caption{Model of SMS data flow}
\label{fig:smsdataflow}
\end{figure}

As seen in figure \ref{fig:chworder}, some overall plan were already in place. The result should be that the \gls{chw}'s should report what they receive whenever they receive any items. This will be registered in the database at the \gls{ndc}. 
This would be straight from a village level to the national level. At the \gls{ndc} there will be a server running \gls{dhis2}, ready to receive data from MTN, the biggest mobile company in Rwanda. The \gls{sms} actually has to go through a \gls{smsc}, before being forwarded to the server at the \gls{ndc}. 
The \gls{dhis2}-instance, from now called "the mobile instnce", will run all the necessary calculations and generate all the results. 
So one of the tasks to be done was to set-up the mobile instance. 
We knew that this would take some time, so in the mean time we sat up a test environment so that we could test our solutions. 


\subsection{Intervention}

\subsubsection{Setting up the Test Environment}
Our initial idea was to set up a \gls{dhis2} instance for testing purposes. This made it possible for us to check if our objectives was in some way already met with the functionality of \gls{dhis2}. We knew that \gls{dhis2} already supported \gls{sms} reporting, but it had never been tested. This was essentially what we did. Configured \gls{dhis2} to support our case. Turned out that objective \#3 and \#4 was already met with just configuring \gls{dhis2}. One thing that we did not think about that became a problem later was the translation of the feedback messages. \gls{chw}'s do not generally speak English, but the local language kinyarwanda. Fortunately, the translation of the messages was possible in the next version of \gls{dhis2}, so the objective was still met. 

\subsubsection{Setting up the mobile instance}
We sat up the \gls{dhis2} instance at the \gls{ndc}. This is server that the \gls{chw}'s will send their reports to. This process was very straight forward. The problem with having our instance running at a different location is that we have less control of our system. Now we have to go through another team to make certain changes to the system. Actually just slows down the whole process. Setting up the mobile instance made our plans more real and allowed us to show our work in real life. 


\subsubsection{User Importer}
The user importer was made in order to import user from a csv file. \gls{dhis2} did not support automatically generating usernames and passwords for bulk users. Therefore we needed a program to do this for us. The down side of this approach is that all the users of the system are not included in the process of creating user accounts. This by passes the \gls{hisp} philosophy of including local users in the system. Users may therefore have user accounts they are not aware of. Making the the users feel less ownership of the system. Despite of this we decided to take this approach.The amount of resources spent on manually register all the users would be to vast. 	

\begin{figure}
\centering
\includegraphics[width=\textwidth]{case/img/userImporterScreenShot}
\caption{Screen Shot of the User Importer}
\label{fig:screenUser}
\end{figure}

The user importer creates user accounts based on firstname, surname, village and phonenumber. After the user accounts are created they are able to send in \gls{sms}-reports based on the village they work from. 

\subsubsection{Re-Supply Algorithm}

\begin{equation}
stk_{n} = stk_{n-1} + rcd_{n} - disp_{n}
\end{equation}

\begin{eqnarray}
reorder_{n} & = & (amc_{n} \cdot 2) - stk_{n} \\
amc_{n} & = & \frac{disp_{n-2} + disp_{n-1} + disp_{n}}{3} \\
disp_{n} & = & stk_{n-1} + rcd_{n} - stk_{n} \\
disp_{n-1} & = & stk_{n-2} + rcd_{n-1} - stk_{n-1} \\
disp_{n-2} & = & stk_{n-3} + rcd_{n-2} - stk_{n-2}
\end{eqnarray}


\begin{description}
\item[$\mathbf{reorder_{n}}$]
This variable represents the quantity of how much is needed at the next re-supply of one village. $n$ in this case represents the last month. If in May, it represents reorder quantity for the end of month of April.
\item[$\mathbf{amc_{n}}$]
Represents the average monthly consumption based on the last 3 months in one village. I in May, that would be the average monthly consumption based on February, March and April.
\item[$\mathbf{disp_{n}}$]
This variable is calculated based on the the values reported and is the number of items distributed by one village during one month.
\item[$\mathbf{stk_{n}}$]
The quantity in stock at the end of the month of one village. Usually reported within 1--5 days into the next month it represents. Stock in April is usually reported between 1st and 5th of May.
\item[$\mathbf{rcd_{n}}$]
This variable is the sum of items received in one village during the month it represents. If a \gls{chw} receives 10 condoms 2nd of April, it should be reported the same day. If a village receives another 10 condoms the 13th of April, that should also be reported the same day it is received. $rcd_{n}$ for April would then be the sum of those values, 20.
\begin{equation}
rcd_{n} = \sum_{k = 1}^{j} rcd_{n,k}
\label{eq:received}
\end{equation}.
A more mathematical description in equation \ref{eq:received}, where j represents the number of days in the month.
\end{description}

\subsubsection{The Essential Predictore}
In order to generate the threshold values to send reminders from we chose to make a small application to run the algorithm. The application updates the database directly. The funny thing is that this was the only way we knew how to get our result, despite the support we had. The developer team from Oslo has offer the competence we needed, but due to the time frame we decided that we needed to build this application with SQL and JAVA. The application was kind of split in two. One of the team members at \gls{hmis} had a strong familiarity with databases and I had some JAVA experience. The core of the application was made in SQL, then it was wrapped inside a JAVA-\gls{gui}.

\begin{figure}
\centering
\includegraphics[width=\textwidth]{case/img/essentialPredictoreScreenShot}
\caption{Screen Shot of the Essential Predictore}
\label{fig:essPred}
\end{figure}

The collaboration in this application is something worth taking note of. Neither one of the team member knew exactly what the other was doing. The application was in fact copy-pasted together after making the the \gls{gui} and SQL-functionality separately. The application realizes the re-supply algorithm in a JAVA application and works with \gls{dhis2}. So all the tools from \gls{dhis2} could be taken advantage of. 
