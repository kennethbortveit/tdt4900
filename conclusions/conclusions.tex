\chapter{Conclusions}
Action research are definitely one of the more appropriate methods of doing research in the field of \gls{ict}'s in the developing countries. 
There are to many uncertainties and unexpected events that could disrupt a more strict research approach.
When doing research coming from a country with more developed technology I found that I had a tendency to over simplify tasks.
This lead to a unrealistic expectancy of progress and not being able achieve the right quality of research.
This being a qualitative data based study, there are a need for more quantitative data studies in the literature.
In the introduction I set out to find answer to why \gls{ict}'s have a hard time being introduced when the technology are available.
My answer to this is that we continue to learn from our mistakes.
In order to do this we have to lift the level of \gls{ict}'s in the local community of the developing countries.
The politicians has to have a more general knowledge of \gls{ict}'s in order to make strategic planning and set these plans into action.
In order for \gls{ict}'s to have a sustainable effect it cannot be seen as a external part, but integrated in all levels of the community.
There are language barriers that need to be addressed. 
People that are not yet being accustomed to the official language of both the internet and the countries.
This leads to misspelling and literature being left unused. 
Activities like health care does not stop because English is not the spoken language, so the supporting systems have to adapt to these challenging circumstances.
There needs to be n uplift in technological expertise in the local community of developing countries. 
While technology are presenting much opportunities like Skype and collaboration with different version control systems, it cannot fully replace the way people usually learn. Through social interactions. 
Therefore we have to build on the local expertise in order to really develop. 
Through workshops and courses with focus on social interactions. 
This leads to the importance of social embeddedness of \gls{ict} initiatives.
The development has to take on a local perspective in order to not introduce to much innovation at one time. 
At the same time to avoid the sustainability pitfall, innovations has to be at a large scale.
Leading to a connected agenda and a deep socio-economic change that needs to be politically facilitated in order to succeed.
By successfully introducing \gls{ict}'s at this level the developing countries has the opportunity to become a part of the learning economy and may be able to provide services as a knowledge based economy. 
Infrastructure and education are two key factors that measurable.
By actively committing to improve these areas residents are presented with both the knowledge and opportunities to contribute to one of the ways out of poverty for the developing countries. 
Taking it back to research questions presented in the introduction.

How can we then improve the process of introducing new technology in developing countries?
By lifting the the local knowledge base in the local community. 
It is not enough to simply provide the technology or drop by in order to have sustainable results.
Preferably the knowledge should already be present before making heavy \gls{ict} investments.
Also, focusing on small incremental changes rather than revolutionary leaps. 
Keeping in mind that there is a digital divide that may make big changes seem like small ones.
A focus on large scale implementations will also contribute to more sustainable ones.

What are the necessary actions needed in order to increase the success rates of \gls{ict} initiatives?
By introducing \gls{ict}'s as a socially embedded process we will be able to make sustainable \gls{ict} initiatives. The best way for doing this is by focusing on lifting the level of \gls{ict} infrastructure and availability of technologies in order to provide the opportunities. In parallel there needs to be a sufficient educational system that provides the necessary knowledge in order to take advantage of these systems. Therefore the actions needed are to improve the educational system as well as upgrade the \gls{ict} infrastructure.

Are there any new challenges that have not yet been highlighted as a result of this \gls{ar}-project?
New challenges include a focus on language adaptable systems, political decision power to people with \gls{ict} knowledge and to invest in order to facilitate change.
Additionally there are the basics of operating computers and \gls{ict} hardware.
There are still examples of people who are looking at a computer for the very first time.
Also there are common misspellings with the letters 'l' and 'o', where users are using number 1 and 0 instead, and vice versa.
This highlights the need for a stronger educational background of new users, preferably provided through the educational system. 
