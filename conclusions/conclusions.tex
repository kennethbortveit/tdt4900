\chapter{Conclusions}
Action research are definitely one of the more appropriate methods of doing research in the field of \gls{ict}'s in the developing countries. 
There are to many uncertainties and unexpected events that could disrupt a more strict research approach.
When doing research coming from a country with more developed technology I found that I had a tendency to over-simplify tasks.
This lead to a unrealistic expectancy of progress and not being able achieve the right quality of research.
Although this paper also contributes to the qualitative data-based studies, it should be mentioned that there is a need for more quantitative studies.
In the introduction I set out to find answer to why \gls{ict}'s have a hard time being introduced when the technology are available.
My answer to this is that we have to be better at learning from our mistakes.
It is difficult to transfer knowledge without having the experience to attach it on. 
Oppertunities for applied knowledge to the users of the knowledge has to be presented and then acted upon in order to gain learning.
Learning in my opinion is not just understanding something new, as is a common misconception.
Learning has to lead to behavior change or else it will be just another memory that can be recalled when asked.
Therefore, learning has to be confirmed in practice.



In order to properly acheieve learning in the developing countries we have to to lift the level of \gls{ict} infrastructure in the local community of the developing countries.
This is because there has to be oppertunities present to apply the knowledge.
Politicians has to be more involved in this process and make decisions that facilitates both learning and the use of \gls{ict}'s.
In order for \gls{ict}'s to have a sustainable effect it cannot be seen as a external part, but integrated in all levels of the community.
Language barriers has to be acknowledged. 
There are people that are not yet being accustomed to the official language of both the internet and the countries.
This leads to misspelling and literature being left unused. 
Activities like health care does not stop because English is not the spoken language, so the supporting systems have to adapt to these challenging circumstances.
There needs to be an uplift in technological expertise in the local community of developing countries. 
While technology are presenting much opportunities like Skype and collaboration with different version control systems, it cannot fully replace the way people usually learn. Through social interactions. 
Therefore we have to build on the local expertise through socal events in order to really develop. 
Through workshops and courses with focus on social interactions and practical work. 
This leads to the importance of the social embeddedness discourse of \gls{ict} initiatives.
The development has to take on a local perspective in order to not introduce to much innovation at one time. 
At the same time to avoid the sustainability pitfall, innovations has to be at a large scale.
Combining large scale and social embeddedness will cause  a deep socio-economic change.
Of course there needs to be a political support in order to successfully adjust to the educational and infrastructial requirements.
By successfully introducing \gls{ict}'s at this level the developing countries has the opportunity to become a part of the learning economy and may be able to provide services as a knowledge based economy. 
Infrastructure and education are two key factors that measurable.
By actively committing to improve these areas residents are presented with both the knowledge and opportunities to contribute to one of the ways out of poverty for the developing countries. 

Taking it back to research questions presented in the introduction.

\textbf{How can we then improve the process of introducing new technology in developing countries?}
By lifting the the local knowledge base in the local community. 
It is not enough to simply provide the technology or drop by in order to have sustainable results.
Focusing on small incremental changes rather than revolutionary leaps. 
Keeping in mind that there is a digital divide that may make evolutionary changes actually being revolutionary ones (seen from a industrialized perspective).
Additionally there needs to be a  focus on lifting the level of \gls{ict} infrastructure and availability of technologies in order to provide the opportunities. Making \gls{ict}'s more common. In parallel there needs to be a sufficient educational system that provides the necessary knowledge in order to take advantage of these systems. 

\textbf{What are the necessary steps needed in order to increase the success rates of \gls{ict} initiatives?}
In order to increase the success rate, we need to focus more on a social embeddedness approach rather than transfer and diffusion.
This includes providing an action network. This helps to avoid the sustainability pitfall. In addition, focus on large scale.
This is preferably done with politcal support in order to be able to act. This calls for \gls{ict} awareness in the political system.
A large scale socially embedded \gls{ict} initiative will lead to deep socio-economic changes, and ultimately, a more developed society.

\textbf{Are there any new challenges that have not yet been highlighted as a result of this \gls{ar}-project?}
New challenges include a focus on language adaptable systems, political decision power to people with \gls{ict} knowledge and to invest in order to facilitate change.
Additionally, basic \gls{ict} training are needed.
There are still examples of people who are looking at a computer for the very first time.
For an example; there are common misspellings with the letters 'l' and 'o', where users are using number 1 and 0 instead, and vice versa.
This highlights the need for a stronger educational background of new users, preferably provided through the educational system. 
