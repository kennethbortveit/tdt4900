\chapter{Conclusions}
Action research are definitely one of the more appropriate methods of doing research about \gls{ict}'s in the developing countries. 
There are to many uncertainties and unexpected events that could distrupt a more strict research approach.
When doing research coming from a country with more developed technology I found that I had a tendency to over simplify tasks.
This lead to a unrealistic expectancy of progress and not being able achieve the right quality of research.
This being a qualitative data based study, there are a need for more quantitative data studies in the literature.
In the introduction I set out to find answer to why \gls{ict}'s have a hard time being introduced when the technology are available.
And how can we increase the success rates and improve the terms of conditions?
My answer to this is that we continue to learn from our mistakes.
There are language barriers that need to be addressed. 
People that are not yet being accustomed to the official language of both the internet and the countries.
This leads to misspelling and literature being left unused. 
Activities like health care does not stop because English is not the spoken language, so the supporting systems have to adapt to these challenging circumstances.
An uplift in technological expertise in the local area has to be lifted. 
While technology are presenting much opportunities like Skype and collaboration with different version control systems, it cannot fully replace the way people usually learn. Through social interactions. 
Therefore we have to build on the local expertise in order to really develop. 
This leads to the importance of social embeddedness of \gls{ict} initiatives.
The development has to take on a local perspective in order to not introduce to much innovation at one time. 
At the same time to avoid the sustainability pitfall, innovations has to be at a large scale.
Leading to a connected agenda and a deep socio-economic change that needs to be politically facilitated in order to succeed.
By successfully introducing \gls{ict}'s at this level the developing countries has the opportunity to become a part of the learning economy and may be able to provide services as a knowledge based economy. 
Infrastructure and education are two key factors that measurable.
By actively committing to improve these areas residents are presented with both the knowledge and opportunities to contribute to one of the ways out of poverty for the developing countries. 



