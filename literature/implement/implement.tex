\chapter{Implementation}
\section{In General}

\section{Updated Information}
\section{Changing Processes}
With information technology comes the great advantage of cutting processes to more effective ones.
In Michael Hammers article \cite{mh:rw} he discusses how information technology can change how people execute their work.
The idea is that computers should not automate existing processes, but rather make room for new and more effective ones to rise.
Hammer's ideas are of some age, but it still highly relevant. 
\textbf{Some examples here}. 
Information passed by paper has the disadvantage of being bound by geography making an organization or system slower due to delivery.
Information technology can bring an organization together and simulate being on the same place, but still being spread out.
This allows for old work processes to be replaced by new ones making the turn-around for each task possibly much faster.

In big organizations the process of swap out old processes can be of very high risk. 
Change in work processes takes the personnel out of their comfort zones and they have to readjust to the new environment. 
The new way of doing things may in a short term perspective seem unproductive.

This calls for leadership with strong vision and determination in order to implement the new processes and reap of the benefits.

\section{Facilitate the transition}


\section{Over complicating}
