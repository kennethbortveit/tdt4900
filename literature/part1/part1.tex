
\section{E-Health}
E-health is defined as:
\begin{quotation}
Use of information and communications technologies in support of health and health-related fields, including health-care services, health surveillance, health literature, and health education, knowledge and research.
\end{quotation}

E-health has many areas of application. Studies have shown that in by using electronic health records it is possible to improve staff productivity, reducing patient wait times, increase staff satisfaction and providing higher quality of data to relevant personnel. Laboratory information managements systems have decreased the time for communicating results and improved the productivity of the laboratory. Pharmacy information systems reduces time to order medications and provides easy access to past information. This is useful for forecasting medication requirements in order to get it at a lower price. Particularly relevant for drug resistant \gls{tb} medications. Also reducing the number of errors. Fingerprint scanners reduced the time to locate records with 74\% and barcode scanners reduced the time by 97\%. The time to track patients lost to follow-up could with patient tracker systems be reduced by 20--50\%. And patient reminder systems can increase attendance with up to 21\%. A cost analysis of data collection systems show that it is possible to save 91\% over a paper based system. 

These findings clearly argue for the implementation and commitment of e-health in developing countries. 

Challenges include:
\begin{itemize}
\item physical environment
\item power
\item networking
\item availabillity of technical staff
\end{itemize}

Further there is a rapid development in the m-health area.
Cellphones and tablets are making the networks previously accessable only to computers available on the move.
In health this means that it is easier to response to crisis, talk to health personell, both as a coleage and patient and generally making health services more available through electronic channels. 
The idea of mhealth is not something completely new. Just think about the primary way of responding to an accident as a citizen. 
Nevertheless, the way software are being integrated with health services today are innovative and needs attention in order to succeed.
Even in the more developed countries of the world there are histories that exemplifies the difficulty of integrating health services with \gls{ict}'s.  

Healthcare is practiced within a highly distributed organizational network.
Patients and clinical data are being sent back and fourth between specialists and generalists.
Today, patients live longer and have a tendency to suffer from chronic deseases. 
This results in that they have to be monitored by several care providers located in different settings.
If not done properly the health service will to a patient seem disorganized and time consuming.
From this problem there has been alot of attention in integrated care. 
Integrated care means that from a patients point of view, the health care should be seamless, although the care are coming from many different places and diciplines. 
\gls{ict}'s are seen as one of the possibilities to make this happen by providing standardized information both faster and more accurate.
\cite{ictcare:winther}
\cite{ehealth:blaya}

\subsection{M-Health}
\begin{quotation}
mHealth is a component of eHealth. To date, no standardized definition of mHealth has been established. 
For the purposes of the survey, the Global Observatory for eHealth (GOe) defined mHealth or mobile 
health as medical and public health practice supported by mobile devices, such as mobile phones, patient 
monitoring devices, personal digital assistants (PDAs), and other wireless devices. 
mHealth involves the use and capitalization on a mobile phone’s core utility of voice and short messaging 
service (SMS) as well as more complex functionalities and applications including general packet radio 
service (GPRS), third and fourth generation mobile telecommunications (3G and 4G systems), global 
positioning system (GPS), and Bluetooth technology.
 Rather than strategic implementation, the emergence of mHealth is occurring in many Member 
States through experimentation with technologies in many health settings. Policy-makers 
and administrators need to have the necessary knowledge to make the transition from pilot 
programmes to strategic large-scale deployments.



Many countries reported up to six mHealth programmes per country.
The survey ranked the adoption of the top mHealth initiatives ranging from Health call centres 
(number 1) to decision support systems (number 14); patterns of adoption were described 
according to World Bank income group and WHO region.
Many of the top six barriers to mHealth implementation related to the need for further 
knowledge and information, such as assessing effectiveness and cost-effectiveness of mHealth 
applications. Other key barriers included conflicting health system priorities, the lack of 
supporting policy, and legal issues.
Although the level of mHealth activity is growing in countries, evaluation of those activities 
by Member States is very low (12\%). Evaluation will need to be incorporated into the project 
management life-cycle to ensure better quality results.
Data security and citizen privacy are areas that require legal and policy attention to ensure that 
mHealth users’ data are properly protected.
Member States will progress further in implementing mHealth if they share global ICT standards 
and architecture. Cooperation in the development of best practices enterprise architecture will 
ensure that data can move more effectively between systems and applications.

\cite{mhealth:who}
\end{quotation}

\begin{quotation}
Implementations of potentially transformative eHealth technologies are currently underway internationally, often with significant impact on national expenditure. England has, for example, invested at least £12.8 billion in a National Programme for Information Technology (NPfIT) for the National Health Service, and the Obama administration in the United States (US) has similarly committed to a US\$38 billion eHealth investment in health care [1]. Such large-scale expenditure has been justified on the grounds that electronic health records (EHRs), picture archiving and communication systems (PACS), electronic prescribing (ePrescribing) and associated computerised provider (or physician) order entry systems (CPOE), and computerised decision support systems (CDSSs) will help address the problems of variable quality and safety in modern health care. However, the scientific basis of such claims—which are repeatedly made and seemingly uncritically accepted—remains to be established

\cite{ehealth:ashly}
\end{quotation}

\subsection{Electronic Health Records}
\gls{ehr} are patient centric medical records. 
These records can include a patients medical history and details of recent care, images, scanned documents and administrative data such as bed management and commission data.
The users of \gls{ehr} are usually clinicians, administrators and patients themselves. 
When medical records are digitized there are the added benefit of interoperability with other systems such as ePrescribing. 
Linking patients with prescriptions. 
The benefits of implementing a \gls{ehr} system are usually related to data storage and data management operations. Increased accessibility, legibility, manipulation, transportation and preservation to name a few. Leading to an overall increase in organizational effeciency and secondary uses. The implementation of an \gls{ehr} system also introduce some risks.
These involve illegitimee access, increased time spent on documentation and make the patient-provider interaction seam less personal.


\subsection{Picture Archiving and Communication Systems}
\gls{pacs} are clinical information systems are used for the acquisition, archival and post-processing distribution of digital images. 
Essentially, a system used to store medical images.
Could be integrated with for an example \gls{ehr}'s.
The benefits of implementing \gls{pacs} relates to improved organisational efficiecy through time savings resulting from improved productivity of radiology services, reduced transit time and improved access. 

\subsection{Computerized Provider Order Entry}
\gls{cpoe} systems have the explicit purpose of transferring orders and return results.
Clinicians typically enter, modify, review and communicate orders through this system and get laboratory test, radiological images and referrals results. 
The potential benefits are of course organizational efficiency as with most systems, but there are some risks involved. Like an increase of time spent on computer activity.



\subsection{ePrescribing}
EPrescribing refers to the system by clinicians to enter modify, review and communicate medication prescriptions. 
The system has the exlipicit purpose of transfering prescriptions from the prescriber to the pharmacy.
These systems has the potential of reducing errors and in turn increase patient safety. 
There are usually an increase in productivity of pharmacists, deacreased turnaround time and more accurate communication between precribers and pharmacy. 


\subsection{Computerised decision support systems}
\gls{cdss} uses clinical and demographic patient information to provide support for decision making by clinicians. The fundamental impact of these systems should be improved clinical decision making.


\section{Groupware}
Groupware development is found among the developers.
When the purchasers buy visible and expensive systems organizational change is likely.
Upper managements is thus likely to commit to helping the system succeed.
Social and political factors that affect the introduction of a system.

\begin{itemize}
\item job redesign 
\item job creation
\item providing training
\item restructuring to work around individuals who will not use the system
\item positive leadership
\end{itemize}

Management is less committed to the less expensive applications or features.
An organization will not restructure itself for each new application the way it does around a major new system. Therefore these systems must adapt to the organization and be fitted into existing work patterns and appealing to everyone who must support it. 

\section{Transition Strategy}
With a transition I talk about taking the system as it is and change it to something new. 
It's the process from old to new. The process of transforming systems or system migration if you will.

Making a transition involves a switch from the old system to the new. There
is the source system, also referred to as the legacy system and the target
system. At one end of the specter we have the Big Bang strategy, were we
taken on an revolutionary approach. A complete new system is developed,
supporting all the required functionality. Then one decideds a time when all
of those involved switches to the new system. This way usually has a high
risk of failure. On the other end of the spectrum we have the evolutionary
approach. Gradually one introduces new functionality, or the same with a
new system, then after the legacy system is not used anymore, one turns off
the switch.

\subsection{Planning and Conducting a Transition Strategy}
There are some predefined methods for conducting a system migration which
also could be used in a transition strategy plan. Remembering that one
moves from a source system into a target system. As mentioned, solutions to
transition problems could be characterized by how revolutionary it is. The
most revolutionary would be redevelopment, followed by migration, maintenance and finally wrapping. One would choose the most appropriate strategy based on the level of risk. 
Like wrapping, one takes almost no risk, since it requires no real change to the system, but instead provides an updated interface for the source system. Although this way is low risk, this could complicate things later on. Making use of wrapping not only slows down the system,
but also makes maintenance more complicated. The most appropriate use of
wrapping is when one wants to make a new Graphical User Interface (GUI).
Like when moving from a text based front-end to a graphical based front-end.
With redevelopment on one end and wrapping on the other, in the middle we
have system migration. This technique allows for a smoother approach while
being able to have control.

\subsection{Migration}
When redevelopment is to risky and wrapping is unsuitable, migration usu-
ally is the best way to go. This allows for both systems to co-exist while
making the transition from one to the other. Migration usually involves
moving an existing system to a new platform. Before making the transition
one has to decide on some basics. Like how one would like to migrate to the
new system. Much of the time is spent on testing the target system. There-
fore it is good practice to not introduce new functionality while migrating to
a new platform. It also makes the testing easier since one could compare with
the old system for output results. New functionality should be introduced
afters the old ones are supported.

\subsubsection{The Cutover}
The cutover is the last step in the migration process. 
Here are three main approaches.

\begin{description}
\item[The cut and run] This is the most revolutionary way of migrating. It is
much like redevelopment and seldom used alone. Once the target sys-
tem is ready on turns of the source system and enable a new feature
rich system.

\item[Phased interoperability] In this strategy incremental steps towards the
target system is used. Replacing functionality over time and slowly
moving towards target system until all functionality is replaced. The
last part of the cutover would be cut and run to some degree.

\item[Parallell operations] In this strategy both systems are running at the same
time. Both source and target system is operational. The target system
is continually tested and only when it's fully trusted, the source system
is disabled.
\end{description}

The cut and run is very simple, but usaually involves high risk. Parallell
operations usually become quite complex, but are fairly safe. Phased interoperability is somewhere in the middle.

\cite{leg:jdbj}


\section{Networks of Action}
Networks of Action is a principle that sets out to remedy the sustainability failure of \gls{is}'s in developing countries. 
A common problem with \gls{is}-initiatives, then ecspecially action research intitiatives, is that the fail to persist over time.
Usually they fail to persist when researchers leave or donor funded projects stops. 
The basic principle is that the scale of the intervention will result in a more robust intervention. 
This changes most common perspective of seeing scale and sustainability as two different problems. 
Rather it views scale as a prerequisite in order to achieve sustainability.
This argues for \gls{is} initiators to form aliances with surrounding network in order to succeed. 
Establishing networks creates opportunities for sharing ecperiences, knowledge  and technology between different actors. 
