
\section{E-Health}
E-health is defined as:
\begin{quotation}
Use of information and communications technologies in support of health and health-related fields, including health-care services, health surveillance, health literature, and health education, knowledge and research.
\end{quotation}

E-health has many areas of application. Studies have shown that by using electronic health records it is possible to improve staff productivity, reducing patient wait times, increase staff satisfaction and providing higher quality of data to relevant personnel. Laboratory information managements systems have decreased the time for communicating results and improved the productivity of the laboratory. Pharmacy information systems reduces time to order medications and provides easy access to past information. This is useful for forecasting medication requirements in order to get it at a lower price. Particularly relevant for drug resistant \gls{tb} medications. Also reducing the number of errors. Fingerprint scanners has reduced the time to locate records with 74\% and barcode scanners by 97\%. The time to track patients lost to follow-up could with patient tracker systems be reduced by 20--50\%. Patient reminder systems can increase health care program attendance with up to 21\%. A cost analysis of data collection systems show that it is possible to save 91\% over a paper based system. 

These findings clearly argue for the implementation and commitment of e-health in developing countries. 
Further there is a rapid development in the m-health area.
Cellphones and tablets are making the networks previously accessible only by computers, available on the move.
From a health perspective, this means that it is easier to response to crisis, talk to health personnel, both as a colleague and patient. 
Generally making health services more available through electronic channels. 
The idea of m-health is not something completely new. Just think about the primary way of responding to an accident as a citizen with your cell phone. 
Nevertheless, the way software are being integrated with health services today are innovative and needs attention in order to succeed.
Even in the more developed countries of the world there are histories that exemplifies the difficulty of integrating health services with \gls{ict}'s.  

Healthcare is practiced within a highly distributed organizational network.
Patients and clinical data are being sent back and fourth between specialists and generalists.
Today, patients live longer and have a tendency to suffer from chronic diseases. 
This results in that they have to be monitored by several care providers located in different settings.
If not done properly, the health service will then, from a patient perspective, seem disorganized and time consuming.
Hence, a lot of attention in the e-health area has been given to integrated care. 
Integrated care means that from a patients point of view, the health care should be seamless. 
This is while the care are coming from many different places and disciplines. 
\gls{ict}'s are seen as one of the possibilities to make this a reality. 
\gls{ict}'s can provide standardized information both faster and more accurate between the different actors in the health sector.

In a global perspective we can see that e-health are on the way.
England has invested £12.8 billion in National Programme for Information Technology for the National Health Service.
The Obama administration in the \gls{usa} has committed to a \$38 billion e-health investment in health care.
Such large investments has been justified because technology like \gls{ehr}'s, \gls{pacs}'s, e-prescribing systems and \gls{cpoe}'s will contribute to solve the problem of variable quality and safety in health care. Some examples of \gls{ict} in the health sector follows.

\cite{ictcare:winther}
\cite{ehealth:blaya}
\cite{ehealth:ashly}

\newpage

\subsection{M-Health}
Mobile health is a sub-field of e-health. It is health practice supported by mobile devices.
These devices include:
\begin{itemize}
\item Mobile phones
\item Patient monitoring devices
\item Personal digital assistants
\end{itemize}
Essentially the use of wireless devices used to assist health care.
This includes taking advantage of technologies like voice and \gls{sms}'s, \gls{gprs}, 3G \& 4G, \gls{gps} and Bluetooth.
The emergence of m-health is occurring through experimenting with technologies in typical health settings. 
In order to fully see the benefits of m-health, there is a need to move towards more strategic implementation. Policy-makers and administrators therefore needs the necessary knowledge to make the transition from pilot program, to strategic large scale deployments.
The top m-health initiatives are call centers.
The main barriers of m-health implementation are related to knowledge and information, such as assessing effectiveness and cost-effectiveness of m-health applications.
Other barriers include conflicting health system priorities, lack of supporting policy and legal issues. 
The level of m-health activity is growing internationally, but evaluation of these activities are are very low. About 12\%. 
Evaluation of the initiatives needs to be an incorporated part of the projects in order to ensure better results.

Data security and citizen privacy are areas that require legal an policy attention to ensure that users data are properly protected.
Different countries will progress faster if they share global \gls{ict} standards and architecture. 
Cooperation in development of best practices will ensure that data can move more effectively between systems and applications.
\cite{mhealth:who}

\subsection{Electronic Health Records}
\gls{ehr} are patient centric medical records. 
These records can include a patients medical history and details of recent care, images, scanned documents and administrative data such as bed management and commission data.
The users of \gls{ehr} are usually clinicians, administrators and patients themselves. 
When medical records are digitized there are the added benefit of interoperability with other systems.
Like with e-prescribing, linking patients with prescriptions. 
The benefits of implementing a \gls{ehr} system are usually related to data storage and data management operations. Increased accessibility, legibility, manipulation, transportation and preservation to name a few. Leading to an overall increase in organizational efficiency and secondary uses. The implementation of an \gls{ehr} system also introduce some risks.
These involve illegitimate access, increased time spent on documentation and make the patient-provider interaction seam less personal.


\subsection{Picture Archiving and Communication Systems}
\gls{pacs} are clinical information systems are used for the acquisition, archival and post-processing distribution of digital images. 
Essentially, a system used to store medical images.
Could be integrated with for an example \gls{ehr}'s.
The benefits of implementing \gls{pacs} relates to improved organizational efficient through time savings resulting from improved productivity of radiology services, reduced transit time and improved access. 

\subsection{Computerized Provider Order Entry}
\gls{cpoe} systems have the explicit purpose of transferring orders and return results.
Clinicians typically enter, modify, review and communicate orders through this system and get laboratory test, radiological images and referral results. 
The potential benefits are of course organizational efficiency as with most systems, but there are some risks involved. Like an increase of time spent on computer activity.

\subsection{E-prescribing}
E-prescribing refers to the system by clinicians to enter modify, review and communicate medication prescriptions. 
The system has the explicit purpose of transferring prescriptions from the prescriber to the pharmacy.
These systems has the potential of reducing errors and in turn increase patient safety. 
There are usually an increase in productivity of pharmacists, decreased turnaround time and more accurate communication between the prescriber and pharmacy. 

\newpage
\subsection{Computerized decision support systems}
\gls{cdss} uses clinical and demographic patient information to provide support for decision making by clinicians. The fundamental impact of these systems should be improved clinical decision making.


This gives an introduction to what e-health is and how it may be used. 
It involves transitioning from the "old way" of doing health care to the "new way".
Especially in developing countries.
In health care there are little room for mistakes.
Therefore the transition has to be carefully planned and closely monitored.



\section{Transition Strategy}
A transition is the process between the old and the new.
The process of transforming systems.
At some point one has to start using the new system and leave the old one behind.
There is the source system, also referred to as the legacy system, and the target
system. The system we want. 
There are different approaches to how one might go about such a task.
At one end of the specter we have the Big Bang strategy, were we
take on an revolutionary approach. A complete new system is developed,
supporting all the required functionality. Then at some point, all actors involved switches to the new target system. This revolutionary approach usually comes with a high risk of failure. 
On the other end of the spectrum we have the evolutionary approach. 
Gradually one introduces new or equal functionality with a new system.
Users of the source system has then more time to adapt to the new system before being introduced to more changes.
There are some categorizations of introducing new systems, from revolutionary to evolutionary;

\begin{description}
\item[\textbf{\textit{Redevelopment}}]\hfill\\
A complete new system is made from the ground up.
Includes all activities from gathering requirements to deploying the system.
Usually involves a lot of risk, but could be the most time effective solution.
\item[\textbf{\textit{Migration}}]\hfill\\
This approach tries to collect the best from both worlds.
A target system are developed while the old one is running.
The target system supports all functionality as the source system, but on a different platform.
One can gradually make use of the new system.
\newpage
\item[\textbf{\textit{Maintenance}}]\hfill\\
One makes local and minor changes to the source system in order to support new functionality, but not enough to be recognized as a new system.
\item[\textbf{\textit{Wrapping}}]\hfill\\
Almost no risk, since it requires no real change to the system. Essentially provides an updated interface for the source system. Although this way is low risk, this could complicate things later on. Making use of wrapping not only slows down the system, but also makes maintenance more complicated. Usually used for adding a graphical user interface.
\end{description} 

\subsection{Migration}
So, when redevelopment is to risky and wrapping is unsuitable, migration usually is the best way to go. This allows for both systems to co-exist while making the transition from one to the other.
Good practice suggests that new functionality are not introduced while migrating to the new platform.
One should wait until all of the functionality of the source system are assured to work on the target system before introducing new functionality.

\subsubsection{The Cut-Over}
The cut-over is the last step in the migration process. 
It is how one chooses to shut down the old system and begin to use the new one.
Here are three main approaches.

\begin{description}
\item[The cut and run] When the target system is ready, one shuts down the source system.
This makes the migration process very similar to redevelopment, with the exception of having source system functionality supported. Main drawback is that itdoes not give the users time to adapt.

\item[Phased interoperability] In this strategy incremental steps towards the target system is used. Replacing functionality over time and slowly moving towards target system until all functionality is replaced with the new system. Users of the system have the added benefit of gradually getting used to the new system. Also there are room for feedback from the users if they are unhappy.

\item[Parallel operations] In this strategy both systems are running at the same time. Both source and target system is operational. 
The target system is continually tested and only when it's fully trusted, the source system is disabled. This way we get all user on-board with the new system. Especially time consuming, but the safest way to go. Very suitable in health care situations where there are no room for errors.\\
\cite{leg:jdbj}
\end{description}
With some background knowledge of general \gls{is}, we move on to look at \gls{ict}'s in developing countries.
