
\section{E-Health}
E-health is defined as:
\begin{quotation}
Use of information and communications technologies in support of health and health-related fields, including health-care services, health surveillance, health literature, and health education, knowledge and research.
\end{quotation}

E-health has many areas of application. Studies have shown that in by using electronic health records it is possible to improve staff productivity, reducing patient wait times, increase staff satisfaction and providing higher quality of data to relevant personnel. Laboratory information managements systems have decreased the time for communicating results and improved the productivity of the laboratory. Pharmacy information systems reduces time to order medications and provides easy access to past information. This is useful for forecasting medication requirements in order to get it at a lower price. Particularly relevant for drug resistant \gls{tb} medications. Also reducing the number of errors. Fingerprint scanners reduced the time to locate records with 74\% and barcode scanners reduced the time by 97\%. The time to track patients lost to follow-up could with patient tracker systems be reduced by 20--50\%. And patient reminder systems can increase attendance with up to 21\%. A cost analysis of data collection systems show that it is possible to save 91\% over a paper based system. 

These findings clearly argue for the implementation and commitment of e-health in developing countries. 

Challenges include:
\begin{itemize}
\item physical environment
\item power
\item networking
\item availabillity of technical staff
\end{itemize}

Further there is a rapid development in the m-health area.
Cellphones and tablets are making the networks previously accessable only to computers available on the move.
In health this means that it is easier to response to crisis, talk to health personell, both as a coleage and patient and generally making health services more available through electronic channels. 
The idea of mhealth is not something completely new. Just think about the primary way of responding to an accident as a citizen. 
Nevertheless, the way software are being integrated with health services today are innovative and needs attention in order to succeed.
Even in the more developed countries of the world there are histories that exemplifies the difficulty of integrating health services with \gls{ict}'s.  

Healthcare is practiced within a highly distributed organizational network.
Patients and clinical data are being sent back and fourth between specialists and generalists.
Today, patients live longer and have a tendency to suffer from chronic deseases. 
This results in that they have to be monitored by several care providers located in different settings.
If not done properly the health service will to a patient seem disorganized and time consuming.
From this problem there has been alot of attention in integrated care. 
Integrated care means that from a patients point of view, the health care should be seamless, although the care are coming from many different places and diciplines. 
\gls{ict}'s are seen as one of the possibilities to make this happen by providing standardized information both faster and more accurate.




\cite{ictcare:winther}
\cite{ehealth:blaya}

\section{Groupware}
Groupware development is found among the developers.
When the purchasers buy visible and expensive systems organizational change is likely.
Upper managements is thus likely to commit to helping the system succeed.
Social and political factors that affect the introduction of a system.

\begin{itemize}
\item job redesign 
\item job creation
\item providing training
\item restructuring to work around individuals who will not use the system
\item positive leadership
\end{itemize}

Management is less committed to the less expensive applications or features.
An organization will not restructure itself for each new application the way it does around a major new system. Therefore these systems must adapt to the organization and be fitted into existing work patterns and appealing to everyone who must support it. 

\section{Transition Strategy}
With a transition I talk about taking the system as it is and change it to something new. 
It's the process from old to new. The process of transforming systems or system migration if you will.

Making a transition involves a switch from the old system to the new. There
is the source system, also referred to as the legacy system and the target
system. At one end of the specter we have the Big Bang strategy, were we
taken on an revolutionary approach. A complete new system is developed,
supporting all the required functionality. Then one decideds a time when all
of those involved switches to the new system. This way usually has a high
risk of failure. On the other end of the spectrum we have the evolutionary
approach. Gradually one introduces new functionality, or the same with a
new system, then after the legacy system is not used anymore, one turns off
the switch.

\subsection{Planning and Conducting a Transition Strategy}
There are some predefined methods for conducting a system migration which
also could be used in a transition strategy plan. Remembering that one
moves from a source system into a target system. As mentioned, solutions to
transition problems could be characterized by how revolutionary it is. The
most revolutionary would be redevelopment, followed by migration, maintenance and finally wrapping. One would choose the most appropriate strategy based on the level of risk. 
Like wrapping, one takes almost no risk, since it requires no real change to the system, but instead provides an updated interface for the source system. Although this way is low risk, this could complicate things later on. Making use of wrapping not only slows down the system,
but also makes maintenance more complicated. The most appropriate use of
wrapping is when one wants to make a new Graphical User Interface (GUI).
Like when moving from a text based front-end to a graphical based front-end.
With redevelopment on one end and wrapping on the other, in the middle we
have system migration. This technique allows for a smoother approach while
being able to have control.

\subsection{Migration}
When redevelopment is to risky and wrapping is unsuitable, migration usu-
ally is the best way to go. This allows for both systems to co-exist while
making the transition from one to the other. Migration usually involves
moving an existing system to a new platform. Before making the transition
one has to decide on some basics. Like how one would like to migrate to the
new system. Much of the time is spent on testing the target system. There-
fore it is good practice to not introduce new functionality while migrating to
a new platform. It also makes the testing easier since one could compare with
the old system for output results. New functionality should be introduced
afters the old ones are supported.

\subsubsection{The Cutover}
The cutover is the last step in the migration process. 
Here are three main approaches.

\begin{description}
\item[The cut and run] This is the most revolutionary way of migrating. It is
much like redevelopment and seldom used alone. Once the target sys-
tem is ready on turns of the source system and enable a new feature
rich system.

\item[Phased interoperability] In this strategy incremental steps towards the
target system is used. Replacing functionality over time and slowly
moving towards target system until all functionality is replaced. The
last part of the cutover would be cut and run to some degree.

\item[Parallell operations] In this strategy both systems are running at the same
time. Both source and target system is operational. The target system
is continually tested and only when it's fully trusted, the source system
is disabled.
\end{description}

The cut and run is very simple, but usaually involves high risk. Parallell
operations usually become quite complex, but are fairly safe. Phased interoperability is somewhere in the middle.

\cite{leg:jdbj}


