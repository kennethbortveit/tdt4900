
\section{Information systems in the learning economy}
\label{infolearnec}
Research in developed countries has shown that learning is a critical factor for economic success.
This is not just critical for firms and industries, but also for regions and countries.
This makes learning a crucial factor for developing countries. 
Learning, being an interactive socially embedded process, are facilitated through the institutional setup or the national innovation system. It's efficiency very much depends on the circumstances.
By making the environment facilitate learning one has the opportunity to increase learning, and in turn, better the economy and technology, making living conditions better. 
Through \gls{ict}, knowledge can be spread from one individual to another. 
By actively making \gls{ict}'s available we also are making knowledge available and facilitating the learning process. 
So with efforts focused on \gls{ict}'s, developing countries are able to learn faster by having greater access to knowledge and as a result having a better economy.
This would actually just take the developing countries to a level already reached by the developing countries. So in order to really have an impact, they would need to have an advantage. 
Japan and \gls{usa} have two very successful, but different approaches to learning. 
The one from \gls{usa} has a focus on explicit knowledge. Here the focus is on reducing tacit knowledge into information with clearly defined processes and facts. A good example at this would be a step-by-step guide in order to learn something new. 
On the other hand Japan has more focus on making tacit knowledge. This is the knowledge that is almost subconscious. You don't necessarily know it as a set of instructions. This learning strategy are often built on the master-apprentice scheme focusing on co-operation, social cohesion and long-term social relationships. 
By acknowledging that there are two types of strategies, there must be a third that combines the best from both. 
Now, since success in the global economy are based on learning there are an opportunity here for the developing countries to not only advance to the level of the developed, but also have an advantage. 

\cite{gedi:erik}


\subsection{IT and Economic Growth}
With IT comes the assumption that it will in some way enable economic growth \cite{ca:ieeg}.
Although it can be said that highly successful businesses is using IT it would be wrong to say that more IT equals more money.
For an example.
The simple view of IT being able to enable economic growth is not enough.
It can however increase productivity in several ways by automating existing processes, but the potential of IT lies in new ways of structuring organizations.
Time and space can be compromised significantly.

In the 1980's there was invested 750 billion \$ in IT \cite{ca:ieeg}, but this only lead to \(0.7\%\) increase in productivity. This was a decrease from the previous decade.
There is findings that suggests that ICT has a positive correlation with productivity. Data from 1983 to 1990 shows this for eleven Asia pacific countries \cite{ca:ieeg}. May be a necessity in order to take part in the global economy and making it possible to trade. IT can also directly affect how organizations structure themselves by introducing new ways of working and increasing productivity.
ICT should be used is withing the organization to enable better work processes, not automatize existing processes \cite{mh:rw}. 
