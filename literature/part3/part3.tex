\subsection{IT and Economic Growth}
With IT comes the assumption that it will in some way enable economic growth \cite{ca:ieeg}.
Although it can be said that highly successful businesses are using \gls{it} it would be incorrect to say that more \gls{it} equals more money.
It can however increase productivity in several ways by automating existing processes, but the potential of \gls{it} lies in new ways of structuring organizations.
Time and space can be compromised significantly.

There are findings that suggests that ICT has a positive correlation with productivity. Data from 1983 to 1990 shows this for eleven Asia pacific countries \cite{ca:ieeg}. 
May be a necessity in order to take part in the global economy and making it possible to trade. 
\gls{it} can also directly affect how organizations structure themselves by introducing new ways of working.

\subsection{Information systems in the learning economy}
\label{infolearnec}
In order for developing countries to take advantage of \gls{is}'s, they must learn how.
Research in developed countries has shown that learning is a critical factor for economic success.
This is not just critical for firms and industries, but also for regions and countries.
This makes learning a crucial factor for developing countries. 
Learning, being an interactive socially embedded process, are facilitated through the institutional setup or the national innovation system. It's efficiency very much depends on the circumstances.
By making the environment facilitate learning one has the opportunity to increase learning, and in turn, better the economy and technology, making living conditions better. 
Through \gls{ict}, knowledge can be spread from one individual to another. 
By actively making \gls{ict}'s available we also are making knowledge available and facilitating the learning process. 
So with efforts focused on \gls{ict}'s, developing countries are able to learn faster by having better access to knowledge and as a result, having a better economy.
This would actually just take the developing countries to a level already reached by the developing countries. 
So in order to really have an impact, they would need to have an advantage. 
Japan and \gls{usa} have two very successful, but different approaches to learning. 
The one from \gls{usa} has a focus on explicit knowledge. 
Here the focus is on reducing tacit knowledge into information with clearly defined processes and facts. 
A good example at this would be a step-by-step guide in order to learn something new. 
On the other hand Japan has more focus on making tacit knowledge. 
This is the knowledge that is almost subconscious. 
You don't necessarily know it as a set of instructions. 
This learning strategy are often built on the master-apprentice scheme focusing on co-operation, social cohesion and long-term social relationships. 
By acknowledging that there are two types of strategies, there must be a third that combines the best from both. 
Now, since success in the global economy are based on learning there are an opportunity here for the developing countries to not only advance to the level of the developed, but also have an advantage. 

\cite{gedi:erik}





\section{Outsourcing}
Offshoring has it's spring from:

\begin{itemize}
\item globalization of trade in services
\item Software commoditization
\item Wage differentials
\item Business friendly climate 
\item Growth of offshore labor pool
\item Drop in telecom costs
\end{itemize}

China graduates four times more engineers than the \gls{usa} pr. year. 
Before there were a differance in the quality of the engineering program, but the gap has narrowed. 
The talent was always there, but before those with talent would emigrate to industrialized countries. 
With globalization of trade in services, anyone can tap into their services from anywhere. 
With low telecom prizes, low wages and software commoditization industrialized countries are able to offshore their software activities more or less. 

The wage factor are the most dominant factor for off-shoring.
The global software work phenomenon is not the first of it's kind, but it differs in that it delivers a service rather than a specific product. 
It is well known that manufactoring and production are often moved to other low-wage countries.
Parts of software development has now become such a commodity that firms from industrialized countries are able to outsource these tasks and keep the more high-level activity for themselves. 

A useful way of understanding this context is via Vernon's \textit{international product cycle}.
\begin{description}
\item[\textit{Stage 1:}]\hfill
A new product begin with highly skilled entrepreneural activities, typically in industrialized nations.
\item[\textit{Stage 2:}]\hfill
Production begin to shift offshore via investments in low-wage nations.
\item[\textit{Stage 3:}]\hfill
As the product standardizes, it is mass produced with cheap, low-skilled labor.
\end{description}


Software has areas in all three stages. The high-level activities stay in stage 1 while being prepared as routine tasks of best practice, then moved towards stage 3 through stage 2.

Global Software Work opens up a market that are very different from others.
The developing countries are here able compete under very different circumstances.
Were the developed countries has to deal with high salaries, the developing countries can benefit from having lower salaries and compete on cost. 
This makes the market highly dependent on the knowledge competencies. 
As discussed in section \ref{infolearnec} a countries ability to learn has a great impact for developing the economy. 
Access to knowledge intuitively has a way of speeding up the learning process. And the most efficient way of getting to knowledge is through \gls{ict}'s. 
Having the opportunity to compete on knowledge competencies can pace the way for developing countries. 
By focusing on learning the developing countries of the world are able to enter the market of \gls{ict}'s with an advantage. 
Policy makers in charge of economic growth and infrastructure should therefore recognize this and facilitate both the learning process and the exportation of services. 
By focusing on this area of expertise development in other areas of industry are likely.
Having a highly developed \gls{ict} infrastructure is likely to have spillover effects on the domestic services and production. 
Making opportunities for even new innovations. 
History has shown that there is a link between fortunes of the developed countries and the developing. 
Rapid upgrades in \gls{ict}'s have reduced the costs and increased the scope of operations all over the world. 

A number of developing countries have nurtured software and \gls{ict} services industries able to compete in the global market, India being the most successful. 
If developing countries are become able to offer services through outsourcing, there may also be an increase to services offered to the domestic organizations. 
This in turn will have an impact on the overall developments of the country or region. 
It is hard to imagine that there are little technological innovations in a place that are among the top exporters of software. 
The spill over effect may result in local organizations running better, and in this way offering better possibilities in the other fields as well.
Having some theoretical foundation, we can move over to describe the case context.
\cite{ca:isdc}
\cite{sbs:gio}
\cite{offit:paan}
