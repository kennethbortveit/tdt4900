\chapter{Literature Review}

\section{Information and Communication Technologies in Developing Countries}

Common problems that concerns the \gls{is} in developing countries are:
\begin{itemize}
\item Scarce resourses
\item Little technology
\item Missing skills
\end{itemize}

\subsection{Discourses}
\label{subsec:discourses}
Chrisanthi Avgerou has presented three common themes that describes \gls{is} innovation. 
\begin{description}
\item[Discourse 1]
	The first discourse assumes that \gls{is} innovation in developing countries is achieved by emulating organizational structure an technology in more developed countries. 
\item[Discourse 2]
	The second discourse assumes that \gls{is} innovation is achieved by first analyzing the local situation and based on this make techno-organizational change that fits.
\item[Discourse 3]
	The last discourse is mostly concerned with creating possibilities for improvement of life conditions. It focuses on how \gls{is} can be used to facilitate deep socio-economic change.
\end{description}

\cite{ca:isdc}

\subsection{Success and Failure}
\label{successandfailure}
\begin{description}
\item[Total Failure] An initiative that never is implemented or abandoned immediately after implementation.
\item[Partial Failure] An initiative where major goals are unattained or where there are significant undesirable outcomes.
\item[Success] An initiative where major goals are attained for most stakeholders and there are no significant undesirable outcomes.
\end{description}



\cite{rh:isdc}


\section{Digital Divide}

\section{E-Health}



\cite{ehealth:blaya}
