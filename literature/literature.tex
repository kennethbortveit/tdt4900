\chapter{Literature Review}

\section{Information and Communication Technologies in Developing Countries}
It is important to take notice of what kind of development \gls{ict}'s are supposed to support.
And in this notion make conclusions and measures based on that.
Like increase a countries competitiveness in the global free market.




Common problems that concerns the \gls{is} in developing countries are:
\begin{itemize}
\item Scarce resourses
\item Little technology
\item Missing skills
\end{itemize}

\subsection{Discourses}
\label{subsec:discourses}
\begin{itemize}
\item The pace and direction is set by the advanced economies in the world, North America and Europe. \cite{ca:isdc}
\item An increasing number of studies in developing countries in Africa, Asia and Latin America.
\item Developing countries are highlighting new topics like national culture, global politics, provision of \gls{ict} resources for a community.
\item 
\end{itemize}
\begin{description}
\item[\textit{Transfer and Diffusion Discourse}]
	The Transfer and Diffusion discourse assumes that \gls{is}-innovations in developing countries are achieved by transferring technology and organizational structure from more advanced countries. Much like, "if it works here, it should work there". In order to succeed the receiving part should try to emulate what is being done in the more developed countries. Of course \gls{is} these ideas of best practice are somewhat adapted to fit their new context, but the underlying assumption is that the transferred methods result in the same outcomes.
\item[\textit{Social Embeddednes Discourse}]
	This discourse assumes that \gls{is} innovation is about creating new techno-organizational structures given a local social context. The new structures are built on the already existing structures and are a locally socially constructed course of action. The problems are seen from a local perspective and hence the solutions has to be an integrated part.
\item[\textit{Transformative Discourse}]
	The last discourse is mostly concerned with creating possibilities for improvement of life conditions. It focuses on how \gls{is} can be used to facilitate deep socio-economic change.
	The social embeddednes discourse takes the local context into consideration, but the transformative takes it one step further and includes politics, economics and social conditions. 
\end{description}

The transformative discourse raises more explicitly the strategic issues in the development struggle.

A distinctive feature of \gls{is} research in developing countries is that it puts focus on e-governments, free and open software and the development of community resources intended to overcome the digital divide.

Issues that received attention in \gls{isdc}.
\begin{itemize}
\item \gls{is}-failure
\item Outsourcing
\item The strategic role of \gls{ict}
\end{itemize}

\cite{ca:isdc}

\subsection{Success and Failure of ISDC}
\label{successandfailure}

Are there any evidence suggesting that there are more \gls{is}-failures in developing countries?




In the \gls{isdc} literature the is an anxiety about failure, and not without reason.
Compared to other settings there are additional pressures.
The need to catch up with the rest of the world, high opportunity costs and over optimistic expectations.




Categories of failure:
\begin{itemize}
\item Scalability failure because of waning political support, technological complexity, human resource capacity.
\item Sustainability failure because of starvation of \gls{is} resource, loose political commitment, poorly maintained. This could be tracked down to foreign aid-donors that neglected the development of local technological capabilities. The remedy is of course to look at \gls{is} as the socially embedded instead of transfer and diffusion. The \gls{is} project needs to be an integrated part of organizational practices, secure the required financial and knowledge based resources and political commitment in order to succeed.
\item Assimilation in dysfunctional organizational processes, meaning that an already broken system cannot be fixed with facilitating the broken system with \gls{is}.
\end{itemize}





\begin{description}
\item[Total Failure] An initiative that never is implemented or abandoned immediately after implementation.
\item[Partial Failure] An initiative where major goals are unattained or where there are significant undesirable outcomes.
\item[Success] An initiative where major goals are attained for most stakeholders and there are no significant undesirable outcomes.
\end{description}




\begin{figure}
\centering
\includegraphics[width=\textwidth]{literature/img/failChart}
\caption{Diagram of Success and Failures in Industrialized Countries}
\label{fig:failchart}
\end{figure}

In industrialized countries there is an estimate of $\frac{12}{60}$ to $\frac{15}{60}$ fall into the category of total failure; something like $\frac{20}{60}$ to $\frac{36}{60}$ fall in the partial failure category; and lastly the $\frac{9}{60}$ to $\frac{28}{60}$ are successes. See figure \ref{fig:failchart}.

For practical reasons like lack of technical and human infrastructure developing countries should be performing worse than industrialized countries.
The evidence base is not that strong due to lack of literature in general, evaluation and focus on case studies, but it generally points in one direction. High rates of \gls{is}-failure.

Failure can be viewed as something positive. 
Like in a learning process, but it cannot be overlooked that continually failing keeps the under developed countries on the wrong side of the digital divide. 

Organizational change can be viewed as a process between two states.
The current system and the future.
The future system would represent the design of a \gls{is} system that are to be implemented.
And the current system as a description of how the system is.
Between these models there will need to be some change in several dimensions like processes, people, structure and technology in order to reach the a state of the future system. This design-actuality gap usually represents how much risk are involved from moving to a design. 

The actuality of the industialized countries and developing countries have in themselves gaps.
The design-actuality gap might not be the same for the solutions used in industrialized countries and developing countries. 

\cite{rh:isdc}
\cite{ca:isdc}


\section{Outsourcing}
A number of developing countries have nurtured software and \gls{ict} services industries able to compete in the global market, India being the most successful. 
Factors that account for success in the global market include technology and project management skills, copyright legislation and government industrial policy. By making an effort to outsource, there may be also be an increase to services offered to the domestic organizations. This in turn will have an impact on the overall developments of the country or region. It is hard to imagine that there are little technological use in a place that are among the top exporters of software. The spill over effect may results in local organizations running better, and this way offering better possibilities in the other fields as well.

\cite{ca:isdc}






\section{Digital Divide}

\section{E-Health}



\cite{ehealth:blaya}
