\begin{abstract}


Firstly we explore the current status of this area in the research community. 
E-health gives us an introduction to why we are trying to introduce information and communication technology's (ICT's) in order to improve health care. 
Work in transition strategies are relevant in the sense that these challenges have been met before when trying to upgrade current systems. 
Introducing ICT's in developing countries has been attempted before with little success, but have resulted in experiences that can provide us with guidelines of how to avoid the same result. 

Further there is another motivation of introducing ICT's in the developing countries that relates to better the economy of developing countries which introduces the "learning economy". 

Then move on to more broad contextual description of the case in order to get a sense of that there are interconnected dependencies and we are not dealing with a isolated situation. 
Before introducing the primary result of the study I try to elaborate on the method used.
With the case in context and an understanding of the method used I introduce the case as seen from my perspective. 

It builds on qualitative data collected from collaborating with professionals at the Ministry of Health in Rwanda. 
Combining research with everyday practice as is the main purpose of action research. 
In action research there are interventions that will have an effect on the problem situation.
These interventions lead to the development of two applications that also are a part of this research project. 
The experiences from this case study are therefore useful since the interventions of the project lead to concrete products and therefore successfully introduced ICT products in the health sector. 



In my opinion, the need to act on ICT's as a way of improving health services in the developing countries are long overdue. 





\end{abstract}