\chapter{Discussion}

\section{Evaluation}
In terms of meeting the objectives described in \ref{sec:objectives} there are still some work to do.
In the time frame of this case we did not manage to make a system that was operable in a real scenario. 
This was largely due to delays in the system of bureaucracy. 
It cannot be stated clearly enough how this effects the development cycle of this project.
This system has been planned for a long time. 
The first time we initiated the request for the \gls{smpp} protocol was at least 4 months prior to this particular case study. Adding 2 months during the case study we end up at 6 months of waiting.
I will not go into details about the political decision making, but in order to explain why the project was mainly realized in a test environment is because of one signature that never made it to the paper.
Because of this fact, I will continue to evaluate the project based on what we where able to realize in the test environment.

\subsection{Objectives}
The first objective (see \ref{desc:objectiveone}) was clearly over simplified in the beginning.
\gls{dhis2} supports both sending notifications based on rules. The main problem was to generate the values the rules should use as thresholds automatically. So the objective should rather have stated that we should be able to integrate a customized algorithm that could use data from \gls{dhis2} as input. The output should then be available in \gls{dhis2} for presentation and further use. 
With "The Essential Predictore" we were able to do just this. The plan was to then use \gls{dhis2} to send notifications based on a functionality called validation rules.
We did not have the time to figure out if this was possible, but was told by the support team that it was very likely. 

The second objective (section \ref{desc:objectivetwo}) was to send a reminder if a \gls{sms} report was late. Our initial idea was to use the validation rules in order to check if the report was sent. Like the first objective, we did not get the time to work on the validation rules and therefore were not able to implement this functionality.

The third and fourth objectives (section \ref{desc:objectivethree} \ref{desc:objectivefour}) were realized in \gls{dhis2}. After configuring \gls{dhis2} we were able to both send and receive \gls{sms} in the test environment. This functionality was also presented to the \gls{chd}.

In terms of success and failure as discussed in \ref{successandfailure} we can categorize the objectives. Assuming that the test environment is suitable to simulate reality, we have; 
\begin{table}
\centering
\begin{tabular}{p{2cm} p{3cm} p{6cm}}
\textbf{Objective} & \textbf{Category} & \textbf{Comment} \\
\hline
\hline
\#1 
& Partial-failure 
& We did not meet the main objective, but did clear out much of the work in order to meet the objective in the future.
\\
\hline
\#2 
& Total-failure 
& We did not meet the main objective in the time frame set for the case.
\\
\hline
\#3 
& Success 
& The main objectives were met, although only in the test environment.
\\
\hline
\#4 
& Success 
& The main objectives were met, although only in the test environment.
\\
\hline
\end{tabular}
\caption{Objective Evaluation}
\label{tab:obev}
\end{table}


\section{Reflection}
The \gls{hmis} team had not only \gls{clmis} to consider during the case study.
Several projects including data quality, malaria, moving servers to the national cloud were running alongside the \gls{clmis}-project. This resulted in divided attention of the team.
Having a positive impact on the collective productivity, the divided attention also results in having to continually update team members. As a result, more meetings are needed and more presentations and more time spent. 
\subsection{Language}
Language barriers are common when collaborating across borders. But in this study we found that there are also symbol barriers.
Common misspellings were between the symbol for the letter 'l' and the number '1'.
Also, between the letter 'o' and number '0'. 
These were pitfalls easy to avoid, but hard to take notice of.
This was mainly resulted by the lack of schooling of the \gls{chw}'s. 
In Rwanda there are currently three languages being spoken. English, French and Kinyarwanda.
English is the working language, but both one cannot take for granted that everybody speaks it.
This makes automatic feedback an issue. Developers may not be aware that the official spoken language is are not spoken by all inhabitants.
\gls{dhis2} has taken some actions to make all their messages customizable and support for customized feedback messages are just around the corner.

\subsection{Programming}
I took notice of the convenience of being able to program the software that we used.
\gls{dhis2} is an open source software with frameworks based on JAVA. 
Being able to customize \gls{dhis2} on this level is essential to meet the clients and users needs.
For now \gls{hisp} has a developer team located in Oslo that is supporting the users requirements.
\gls{dhis2} is active in over forty countries. 
Clearly some local customization is in order. 
I would propose that every team that are using \gls{dhis2} had some employee that are able make  \gls{dhis2} applications. 

\subsection{Power supply and Internet}
Often taken for granted in the developed countries are stable internet and power supply.
Every other day one could experience power cuts that lasted between 5 and  10 minutes.
While not being a critical issue in these short terms, it did affect the productivity. 
Routers rebooting and interconnections resulting in downtime on servers are not easy to work with.


\subsection{Creative Use of DHIS2}
\gls{dhis2} is developed to be used a certain way. All data should be located in the same database basicly being manipulated through the user interface.
Our solutions to our problems did in some way circumvent this.
By connecting directly to the database we were able to implement the algorithm and create bulk users.
This may create some issues in the future. 
Like when creating bulk users, the \gls{chw}'s are not being part of the process. 
This may lead to unwanted outcomes on bigger perspective. 
The \gls{dhis2} protocol for doing this is that someone in charge will register users in their area of responsibility.
By bypassing the user involvement we are loosing the \gls{hisp} characteristic of social-embeddednes. 


Another example of creative use of \gls{dhis2} is that instead of using one database several are used for different topics. Then, in order to have interoperability between instances, certain dataelements are transferred between the servers. 

\section{Rwandas ICT Transformation}
The government of Rwanda has lately been wanting to become a knowledge based economy. 
This means that they want to trade knowledge for other kind of resources. 
The topic of interest is \gls{ict}'s.
In section \ref{subsec:discourses} we talked about how \gls{ict} could be categorized into three different discourses of seeing \gls{it}-innovations. 
The Rwandans perspective clearly have some similarities with the third discourse, \textit{transformative \gls{isdc}}. 
Since Rwanda mainly exports coffee and tea there will be some significant changes in the way the country is both social and economical. 
With the goal of becoming the \gls{it} capital of Africa raises issues including social, political and economic issues. 
There are about 90\% of the population that works in the agricultural sector where their main export are tea and coffee. 
Switching from agricultural to an \gls{ict} based economy will call for a large scale deep socioeconomic change.

Into this vision of Rwanda becoming a \gls{ict}/knowledge based economy, \gls{hisp} fits right in.
\gls{hisp}'s view of \gls{it} innovation is more like the social embeddedness discourse.
Since these two discourses are much alike they also can co-exist. While the government is focusing the the socio-economic process the country needs to go through, it makes room for organizations like \gls{hisp} to focus on the social-embeddedness of \gls{ict} innovations.

With becoming a knowledgebased economy a developing country can have great benefit from initiating projects presented in this case.
Rwanda has the oppertunity to develop systems that are not yet present in the more developed countries of the world.
They have the advantage of not being to reliant on legacy systems. 
This has the potential of making room for more advanced systems to be implemented. 
The knowledge and experiences that are generated by doing this can further be used to offer services that are not yet present in the developed countries. 
Particularly the services developed would be of benefit to other developing countries and thusly making the oppertunity for outsourcing possible. 
This of course requires a national effort in order to supply the much needed infrastructue, technology and learning oppertunities needed to take advantage. 
Also there are some businesses in the developed countries that are seeing developing countries as untapped markeds.
Of course, some ethical problems here, but since there are businesses that see potential, local goverments should take this opertunity to take this initiative for themselves. Either by supporting local businesses or act on their own. 
A great deal of efforts have been done by relaunching services offered in the developed parts of the world in new areas were this type of services are seen as novelty.
If there is a time when developing countries should sieze the opertunity to get on the technological train, it is now. 
Before outside actors take to much market shares. 



\cite{overview:rdb}
\cite{rw:snl}

\section{ICT to Facilitate Health Services}
While talking to one of the doctors working as an volunteer in Rwanda she said:
\begin{quotation}
"It's hard to deliver the required services when we don't have the supplies needed in order to give them."
\end{quotation}
Another doctor said that he had over 50 patients a day. 
\begin{quotation}
"I can't even treat a common cold with this amount of time"
\end{quotation}
Their was an estimate that there was about 1700 doctors in Rwanda and just under 1000 of these were practicing.
A rough estimate of 12000 patients pr. doctor. 
Unreal, but very near reality in these type of curcumstances. 
This makes the community health workers an essential and vital part of the health system. 
In order to have these workers provide some kind of treatment they need supplies.
This is what the \gls{clmis} are supposed to facilitate. 
These systems are also known as pharmacy information systems. 
Such systems have the potential of reducing time to order medications and necessary supplies needed in order provide basic health care.
Current systems implemented in Rwanda are paper based. 
Intuitively one can argue that an information system can greatly improve such a system. 
While \gls{dhis2} are not currently designed to support all the necessary requirements of a pharmacy information system it has the potential of being able to be. 
It is open source and are currently supporting the local users to develop applications.
Here would the local community of developing countries like Rwanda greatly benefit from taking action.
By developing their own application they are able to contribute to their body of knowledge, to their own health \gls{is} and in the future being able to offer these services to others.
By implementing a pharmacy information system there are also the added benefit of getting medications at a lower price.
The pharmacy information systems makes it possible to forecast medication requirements and therefore one might be able to order in bigger quantities. 

\section{Avoiding developing pitfalls}
There are common pitfalls related to \gls{ict}-projects in developing countries. 
In this case by including the \gls{hisp}-network the scale of the \gls{dhis2} project is at a large scale. 
The \gls{hmis} team in Rwanda are cooperating with eats neighboring countries trying to define common indicators for the east Africa region.
By scale we have effectevly bypassed the sustainability fall. \gls{hisp} provides a large network that facilitates the process of continued action to meet the objectives set.
Near the end of the \gls{ar}-project the \gls{hmis} team participated in a East-African workshop were experiences with health information systems were shared.
In the case there was a lack of development capabilities in the team. 
One of the key features with open source software like \gls{dhis2} is that of being able to customize the software.
This advantage was not exploited to the fullest. 
With a key team member that focuses on getting familiar with the source code and even being able to contribute to the further improvement of the software, the team capability would be greatly increased. 
This takes us to one of the common pitfalls \gls{ict} failures in developing countries.
The lack of technological expertise or knowledge. 
During the case there were some discussion around the subject. 
Team members would argue that this was common wish among them, but due to attending meetings and being pulled in other directions there simply were not the time during business hours. 
This would then not be done within business hours, but up to the aspiring individuals to learn on their free time.

\section{Importance of cell phones}
Global internet penetration are estimated to be around $34.5\%$, the global mobile subscribers are estimated to be $95.5\%$ per. hundred people. 
In Africa the internet reaches $15.6\%$ of the people while the mobile should be able to reach about $69.3\%$ per. hundred.
With the introduction of smart phones, there will be a whole new line of services introduced through mobile cell phones.
Since the internet coverage are not that great in the developing countries mobile phones are a technology that effectively can reach most people. 
The introduction of m-health are already being rolled out in the developing nations, and are currently an effective way to offer services in the developing.
There are already systems at work in the developing countries that provides some basic health care services.
In our case we saw that mobile phones can be used to manage stock of supplies and medicines. 
With the introduction of smartphones there is an exiting future for the m-health. 
This will in turn provide \gls{chw}'s with information of best practices and allow for pharmacy \gls{is} to become even more user friendly.
As seen in the case description, the \gls{sms} based interface for the users are troublesome. 
There is a high chance of mispelling and writing the messages takes up much time. 


\cite{internet:stats}
\cite{mobile:think}

\section{Vigor to developers}
From the case  we have that we were not able to proceed as planned because we needed support for a protocol.
This is one of the cases were everything seemed to be on order, but due to critical factors that was out of awereness, we were not able to progress. 
This shows that there may be an organizational decision process that may need to include \gls{ict} representatives in order to achieve effective progress. 
It does not much good to have technology ready, but lack the vigor to act upon it. 
This argues for a perspective like the third discourse discussed in section \ref{subsec:discourses}. 
The people that are in fact developing the system need the right amount of vigor in ordre to implement the changes.

\section{Transition to \gls{ict}}
In developing countries we have a low level of integrated \gls{is}'s compared to west.
Now that the world are being more and more connected technologies are being introduced at such a rapid pace in the developing countries that it results in a revolutionary approch.
This means we are know at a stage were the digital divide are pushing for change.
It should be acknowledge that paper based systems are legacy systems. 
These systems have been by growned accustomed to by the users. 
The technology in the west has been adopted very fast, but it has been countinous process. 
The way that technology are being introduced in the developing countries can be seen an step wise, and by doing this introducing more risk for failure. 
As transition theory suggests, a "big bang" approach introduces more risk, while wrapping introduces the least. 
Wrapping in this sense can be seen as developing countries are just using \gls{ict} to be able to communicate with the other countries and not as an integrated part of their day to day practis. 
The rush to decrease the level of digital divide may therfore be counterproductive. 
By introducing to much innovation at the same time can therefor expalain why \gls{ict} initiatives are failing.
This connects the social embeddedness discourse with system migration. 
We therefore have in optional strategies like phased interoperability and parallell operations that could be applied to decrease the rate of failure in developing countries. 

\section{Investment}
In groupware literature there is found that when purchasers buy expensive visible systems, upper management are likely to support it.
This comes from the desire to get something in return for their investment.
When first purchased suddenly there is a willingness to make way for redesigning jobs, create new positions, provide training and positive leadership. 
There are also cases were work are structured around individuals that will not use the system. 
\cite{groupware:grudin}
In our case this might be a contributing factor for why there was less interest in the attainment of the \gls{smpp} protocol.
\gls{dhis2}, being an open source project, are not charging their users. 
This may lead to less commitment from managers and making it difficult to implement.


\section{Knowledge resources}
The type of technical expertise needed to be able to respond to continually changing requirements goes a little deeper than being able to configure one software. 
In the case the team had alot of knowledge relating to the software was being implemented, but there was a lack of coding expertise.
This makes it difficult to take advantage of the open source software. 
One key feature is that on are able to custumize it as one see fit. 
Thats one of the areas that was missing in the team a developer or a developer team.



